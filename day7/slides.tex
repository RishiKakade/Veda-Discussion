\documentclass[pdf]{beamer}
\mode<presentation>{}
\usetheme{Rochester}
\usecolortheme{whale}
\beamertemplatenavigationsymbolsempty

\usepackage{fontspec}
\setmainfont[ItalicFont={brilli.ttf}, BoldFont={brillb.ttf}]{brill.ttf}
\usefonttheme{serif}
\usepackage{microtype}
\usepackage{multicol}

\newcommand{\Subitem}[1]{{\setlength\itemindent{12pt} \item[-] #1}}
\newcommand{\Subsubitem}[1]{{\setlength\itemindent{24pt} \item[○] #1}}
\newcommand{\Subsubsubitem}[1]{{\setlength\itemindent{36pt} \item[-] #1}}
\newcommand{\Subsubsubsubitem}[1]{{\setlength\itemindent{48pt} \item[○] #1}}

\title{Meeting 7: Áhiṁsā}
\subtitle{Nonviolence \& vegetarianism in earlier \& later Hinduism}
\author{Nikhil Surya Dwibhashyam, Agni Purohit}
\date{10 April 2022}

\begin{document}

\frame{\titlepage}

\begin{frame} \frametitle{WhatsApp group}
\begin{center}
	\href{https://chat.whatsapp.com/IXCQEkhfrcwHI7CNY8Fgat}{https://chat.whatsapp.com/IXCQEkhfrcwHI7CNY8Fgat}
\end{center}
\end{frame}

\begin{frame} \frametitle{Next week}
\begin{center}
	We'll be zooming in on R̥V 10.129 for the next two weeks, so there won't be slides.
\end{center}
\end{frame}

\begin{frame}[label=questions] \frametitle{Some questions to consider}
\begin{itemize}
	\item What are the Vedic moral prescriptions on violence in general?
	\Subitem What is the special role of the cow in the Vedic religion?
	\item In what contexts are these prescriptions overridden by ritualistic goals?
	\Subitem Is it permissible or an obligation to sacrifice and eat animals?
	\item To what extent is this Vedic mindset present in Epic Hinduism?
	\Subitem in \emph{modern} Hinduism?
	\item We'll return to these at the end!
\end{itemize}
\end{frame}

\begin{frame} \frametitle{What is áhiṁsā?}
\begin{itemize}
	\item Álpʰa-privative endocentric (tat-púruṣa) from $\surd$\textit{hiṁs}
	\item First found in Brā́hmaṇa literature
	\Subitem Non-privative appears later, probably back-formed
    \item Post-Vedic: wife of Dʰárma, development of the Ṡā́kta tradition
	\item $\surd$\textit{hiṁs} is much older: e.g.~\textit{hiṁsrá} (of Agní) in R̥V
	\item Thus: non-harming (where harm can be good or bad)
	\Subitem Harming whom?
	\Subsubitem Men
	\Subsubitem Kine
	\Subsubitem Other animals
\end{itemize}
\end{frame}

\begin{frame} \frametitle{Áhiṁsā towards men}
\begin{itemize}
	\item Obviously: killing Ā́rya-s not condoned in general
	\Subitem Two acceptable contexts:
	\Subsubitem Battle (e.g. Ten Kings, \emph{between} Ā́rya-s)
	\Subsubitem Human sacrifice (present but rare)
	\item Killing Dásyu-s?
	\item Pacifism? Vedic, Epic, Vedā́nta
\end{itemize}
\end{frame}

\begin{frame} \frametitle{Áhiṁsā towards kine}
\begin{itemize}
	\item Go-medʰá and other rites
	\item Sacrifice of bulls: R̥V 10.86.14
	\item Yet: agʰnyā (f., less commonly m.): synonymous with cow
	\item Synthesis: kine, unlike other animals, are morally akin to men.
	\Subitem Can only be sacrificed in special circumstances
\end{itemize}
\end{frame}

\againframe{questions}

\end{document}
