\documentclass[pdf]{beamer}
\mode<presentation>{}
\usetheme{Rochester}
\usecolortheme{whale}
\beamertemplatenavigationsymbolsempty

\usepackage[ngerman]{babel}
\usepackage{fontspec}
\setmainfont[ItalicFont={brilli.ttf}, BoldFont={brillb.ttf}, BoldItalicFont={brillbi.ttf}]{brill.ttf}
\usefonttheme{serif}
\usepackage{microtype}
\usepackage{multicol}

\newcommand{\Subitem}[1]{{\setlength\itemindent{12pt} \item[-] #1}}
\newcommand{\Subsubitem}[1]{{\setlength\itemindent{24pt} \item[○] #1}}
\newcommand{\Subsubsubitem}[1]{{\setlength\itemindent{36pt} \item[-] #1}}
\newcommand{\Subsubsubsubitem}[1]{{\setlength\itemindent{48pt} \item[○] #1}}

\title{Meeting 23: The Creator}
\subtitle{Prajā́pati in the Vedic literature}
\author{Nikhil Surya Dwibhashyam}
\date{11 September 2022}

\begin{document}

\frame{\titlepage}

\begin{frame} \frametitle{Meeting agenda}
\begin{itemize}
	\item (Re)introduction to discussion group
	\item Today's topic præsentation
	\item Free discussion
\end{itemize}
\end{frame}

\begin{frame} \frametitle{Why a Véda discussion group?}
\begin{itemize}
	\item What is Vedic literature?
	\Subitem Sáṁhitā-s
	\Subitem Brā́hmaṇa-s
	\Subitem Others?
	\item Véda-s vs.~Vedā́nta,  Upaniṣád-s, Purāṇá-s
	\Subitem Later monism (ádvāita) vs.~earlier dualism
	\item Modern (religious, scholarly) fixation upon later literature
\end{itemize}
\end{frame}

\begin{frame} \frametitle{What is there to discuss?}
\begin{itemize}
	\item History \& linguistics
	\item Metaphysics of religion
	\item Exegesis (e.g.~countering western narratives)
	\item Moral principles
	\item Nuances of Ṡrāutá ritual
\end{itemize}
\end{frame}

\begin{frame} \frametitle{Our plans}
\begin{itemize}
	\item Weekly meetings on Sundays at 1200 EDT
	\Subitem Followed by more in-depth literature meeting at 1330 EDT
	\item Different topic/aspect/angle discussed in depth every week
	\item Format:
	\Subitem Præsentation
	\Subitem Free-form discussion
	\item Eventually something more
\end{itemize}
\end{frame}

\begin{frame} \frametitle{WhatsApp group}
\begin{center}
	\href{https://chat.whatsapp.com/IXCQEkhfrcwHI7CNY8Fgat}{https://chat.whatsapp.com/IXCQEkhfrcwHI7CNY8Fgat}
\end{center}
\end{frame}

\begin{frame} \frametitle{Let's introduce ourselves!}
\begin{itemize}
	\item Name
	\item School (if student)
	\item How you found this group
	\item Briefly: background/interest in Hinduism \& Vedic literature
\end{itemize}
\end{frame}

\begin{frame}[label=questions] \frametitle{Some questions to consider}
\begin{itemize}
	\item What is Prajā́pati's role in the Vedic pantheon?
	\Subitem His relationship to other Gods?
	\item What evidence is there that Prajā́pati's role evolved across the Vedic period?
	\item What does Prajā́pati show about the metaphysical nature of the Gods?
	\Subitem their equivalence to one another?
	\item What does Prajā́pati show about the morality of the Gods and the laws that govern Them?
\end{itemize}
\end{frame}

\begin{frame} \frametitle{An introduction to Prajā́pati}
\begin{itemize}
	\item \textit{Prajā́nām páti}: lord of the created (beings)
	\Subitem Or ``lord of creation''
	\item Clear progenitor to later Brahma
	\item Creator of everything in the universe (including the Gods and Ásura-s, e.g.~ṠB 11.1.6.14)
	\item Was first being (ṠB 2.2.4.1)
	\item Creator of new souls of all living things: thus, ongoing role
	\item One King, supreme God
	\Subitem Arguably closest there is in Vedic Hinduism to an Abrahamic conception of God
\end{itemize}
\end{frame}

\begin{frame} \frametitle{Prajā́pati in the R̥g-vedá}
\begin{itemize}
	\item Name of Savitŕ̥ (4.53.2)
	\Subitem Prajā́pati = Savitŕ̥ = Sū́rya = Dív = Índra
	\item Is purified sóma (pávamāna, 9.5.9)
	\item Brings forth children (10.85.43)
	\item Brings auspicious cattle (10.169.4)
	\item Places the seed in the womb (10.184.1)
	\item Perhaps the creator mentioned in 10.129.7
	\item ``Evolution'' in role?
\end{itemize}
\end{frame}

\begin{frame} \frametitle{Prajā́pati as the primordial germ}
\begin{itemize}
	\item Prajā́pati = Hiraṇya-garbʰá: the golden germ
	\Subitem {\textit{Bʰūtásya jātáḥ páti} (R̥V 10.121.1)}
	\Subitem ``O Prajā́pati, thou alone dost understand all these creations, and no one else.'' (R̥V 10.121.10).
	\item Thought to be the creator of the Púruṣa (R̥V 10.90)
	\Subitem But also the Púruṣa Himself
	\item The pseudo-epithet Ká (and ``Egyptian'' associations)
	\Subitem Later an actual epithet, e.g.~TS 1.7.6.6
\end{itemize}
\end{frame}

\begin{frame} \frametitle{Prajā́pati \& the cosmic incest}
\begin{itemize}
	\item \textit{Pitŕ̥}: R̥V 10.61.7
	\Subitem See earlier equivalence
	\Subitem More explicit in MS 4.2.12
	\Subitem Daughter = Uṣás
	\item Punished by Rudrá (ṠB 1.7.4.3, R̥V 1.71.5, \&c.)
	\item Later healed by the Gods (ṠB 1.7.4.4)
	\item Story also found in Purāṇá-s
	\item Meaning of motif?
	\Subitem Gods are not always morally perfect.
	\Subitem Another aspect: Gods as forces of nature
	\Subsubitem Prajā́pati as the (pro)creative urge
\end{itemize}
\end{frame}

\begin{frame} \frametitle{Prajā́pati in ritual}
\begin{itemize}
	\item TS 3.1: supplement to sóma sacrifice
	\item \textit{Prajā́patir akāmayata prajā́ḥ sr̥jey’ éti.}
	\Subitem ``Prajā́pati desired: may I create offspring.''
	\item Describes creation of various creatures
	\item Oblations commonly offered in other sacrifices
	\item Little effect of ``dishonor''
\end{itemize}
\end{frame}

\begin{frame} \frametitle{Prajā́pati in ritual (contd.)}
\begin{itemize}
	\item Shape of altar in agni-cityā́: supposedly shape of Prajā́pati
	\item ``Golden man''
	\Subitem Only instance of quasi-idolatry in Vedic religion, but not really idolatry
	\Subsubitem Also repræsents Agní and the sacrificer (ṠB 7.4.1.15)
	\Subsubitem Same verse: Prajā́pati = Púruṣa
\end{itemize}
\end{frame}

\begin{frame} \frametitle{Prajā́pati(-s) in later Hinduism}
\begin{itemize}
	\item Class of beings (as usual)
	\item Often said to be 21
	\item MBʰ: used to describe Kr̥ṣṇá
	\item As Brahma: fallen into dishonor for several reasons in Purāṇá, including incest
	\item Nonetheless incorporated into the classic trinity of Gods, all derived from the Véda
\end{itemize}
\end{frame}

\begin{frame} \frametitle{References}
\begin{itemize}
	\item Bhattacharji, S. ``Rise of Prajāpti in the Brāhmaṇas.'' \textit{Ann.~Bhandarkar.~Orient.~Res.~Inst.}, \textbf{1983}.
	\item MacDonell, A.~A. \textit{Vedic mythology.} \textbf{1897}.
	\item Gonda, J. \textit{Prajāpati's rise to higher rank.} \textbf{1986}.
	\item Gonda, J. ``Vedic gods and the sacrifice.'' \textit{Numen}, \textbf{1983}.
	\item Gonda, J. ``The popular Prajāpati.'' \textit{Hist.~Relig.}, \textbf{1982}.
	\item Heesterman, J.~C. ``Self-sacrifice in Vedic ritual.'' \textit{Gilgul}, \textbf{1987}.
	\item Joshi, J.~R. ``Prajapati in Vedic mythology \& ritual.'' \textit{Ann.~Bhandarkar.~Orient.~Res.~Inst.}, \textbf{1972}.
\end{itemize}
\end{frame}

\againframe{questions}

\end{document}
