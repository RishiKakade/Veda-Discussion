\documentclass[pdf]{beamer}
\mode<presentation>{}
\usetheme{Rochester}
\usecolortheme{whale}
\beamertemplatenavigationsymbolsempty

\usepackage[ngerman]{babel}
\usepackage{fontspec}
\setmainfont[ItalicFont={brilli.ttf}, BoldFont={brillb.ttf}, BoldItalicFont={brillbi.ttf}]{brill.ttf}
\usefonttheme{serif}
\usepackage{microtype}
\usepackage{multicol}

\newcommand{\Subitem}[1]{{\setlength\itemindent{12pt} \item[-] #1}}
\newcommand{\Subsubitem}[1]{{\setlength\itemindent{24pt} \item[○] #1}}
\newcommand{\Subsubsubitem}[1]{{\setlength\itemindent{36pt} \item[-] #1}}
\newcommand{\Subsubsubsubitem}[1]{{\setlength\itemindent{48pt} \item[○] #1}}

\title{Meeting 17: The Aşvín-s}
\subtitle{Horsemen, doctors, sons of Heaven}
\author{Nikhil Surya Dwibhashyam}
\date{19 June 2022}

\begin{document}

\frame{\titlepage}

\begin{frame} \frametitle{Meeting agenda}
\begin{itemize}
	\item (Re)introduction to discussion group
	\item Today's topic præsentation
	\item Free discussion
\end{itemize}
\end{frame}

\begin{frame} \frametitle{Why a Véda discussion group?}
\begin{itemize}
	\item What is Vedic literature?
	\Subitem Sáṁhitā-s
	\Subitem Brā́hmaṇa-s
	\Subitem Others?
	\item Véda-s vs.~Vedā́nta \& Upaniṣád-s
	\Subitem Later monism (ádvāita) vs.~earlier dualism (dvāitá)
	\item Modern (religious, scholarly) fixation upon later literature
\end{itemize}
\end{frame}

\begin{frame} \frametitle{What is there to discuss?}
\begin{itemize}
	\item History \& linguistics
	\item Metaphysics of religion
	\item Exegesis (e.g.~countering western narratives)
	\item Moral principles
	\item Nuances of Ṡrāutá ritual
\end{itemize}
\end{frame}

\begin{frame} \frametitle{Our plans}
\begin{itemize}
	\item Weekly meetings on Sundays at 1200 EDT
	\Subitem Followed by more in-depth literature meeting at 1330 EDT
	\item Different topic/aspect/angle discussed in depth every week
	\item Format:
	\Subitem Præsentation
	\Subitem Free-form discussion
	\item Eventually something more
\end{itemize}
\end{frame}

\begin{frame} \frametitle{WhatsApp group}
\begin{center}
	\href{https://chat.whatsapp.com/IXCQEkhfrcwHI7CNY8Fgat}{https://chat.whatsapp.com/IXCQEkhfrcwHI7CNY8Fgat}
\end{center}
\end{frame}

\begin{frame} \frametitle{Let's introduce ourselves!}
\begin{itemize}
	\item Name
	\item School (if student)
	\item How you found this group
	\item Briefly: background/interest in Hinduism \& Vedic literature
\end{itemize}
\end{frame}

\begin{frame} \frametitle{Who are the Aşvín-s?}
\begin{itemize}
	\item External view
	\Subitem Etymology straightforward this time: $\surd$\textit{aşvá}
	\Subitem Usually but not always dual
	\Subitem IE: Ašvieniai, Kástōr \& Polydeúkēs, \&c
	\Subsubitem Diff.~parentage
	\Subsubitem Solar associations
	\Subsubitem Oaths \& oath-breaking
\end{itemize}
\end{frame}

\begin{frame} \frametitle{Who are the Aşvín-s?}
\begin{itemize}
	\item Internal view
	\Subitem Most frequent other than Índra, Agní, Sóma
	\Subsubitem 398 mentions in R̥V
	\Subsubitem Over 50 designated hymns in R̥V
	\Subitem Never named separately
	\Subitem Synonymous with (unique) epithet \textit{nā́satya} (not untruthful)
	\Subsubitem Dæmons in Avesta! (Vd.~10.9)
	\Subitem Descendants of the Heavens (\textit{divó nápātā})
	\Subitem Yāska: son of night, son of dawn (Nir.~12.2)
\end{itemize}
\end{frame}

\begin{frame} \frametitle{Vedic descriptions of the Aşvín-s}
\begin{itemize}
	\item Youngest of the gods (TS 7.2.7.2)
	\item Golden (R̥V 8.8.2), bright (R̥V 7.68.1), honey-colored (R̥V 8.26.6)
	\Subitem Lots of other honey imagery
	\item Dancing (R̥V 6.63.5)
	\item Have garland/wreath of flowers (R̥V 10.184.2)
	\item Wondrous \textit{dasrá}, fairly unique
	\item \textit{Rudrá-vartani}, \textit{híraṇya-vartani} (red/bright/golden of path)
	\Subitem Diff.~indig.~interpretation of these \& \textit{dasrá}
\end{itemize}
\end{frame}

\begin{frame} \frametitle{Vedic mythemes about the Aşvín-s}
\begin{itemize}
	\item Sun-skinned chariot (R̥V 8.8.2), IE connection
	\Subitem Accompanied by birds (R̥V 6.63.6)
	\Subsubitem Drawn by them \& other animals? Contradiction?
	\item Gave 100 jars of honey (R̥V 1.117.6)
	\item Connected with travel of Sun in a day (R̥V 3.58.8)
	\Subitem Præcede the Dawn (R̥V 1.34.10)
	\Subitem e.g.~\textit{parijman}
	\item Physicians (R̥V 1.157.6); initially \textit{asomapa}
	\item Helped Índra slay Vr̥trá (R̥V 8.8.9)
	\item Involved in Átri myth?
\end{itemize}
\end{frame}

\begin{frame} \frametitle{The Aşvín-s after the Véda}
\begin{itemize}
	\item Not huge differences as seen with Índra, Rudrá
	\item Descent into obscurity
	\item MB: Cyavana story (also in ṠB)
	\item Nakula \& Sahadeva
\end{itemize}
\end{frame}

\begin{frame} \frametitle{The Aşvín-s outside Hinduism}
\begin{center}
	Anything?
\end{center}
\end{frame}

\begin{frame} \frametitle{Some references}
\begin{itemize}
	\item Kazanas, N.~D. ``Indo-European deities \& the R̥g-vedá.'' \textit{J.~Indoeur.~Stud.}, \textbf{2001}.
	\item MacDonell, A.~A. ``Vedic Mythology.'' \textit{Grundr.~Indoar.~Philol.~Altertumskd.}, \textbf{1897}.
\end{itemize}
\end{frame}

\end{document}
