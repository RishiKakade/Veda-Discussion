\documentclass[pdf]{beamer}
\mode<presentation>{}
\usetheme{Rochester}
\usecolortheme{whale}
\beamertemplatenavigationsymbolsempty

\usepackage[ngerman]{babel}
\usepackage{fontspec}
\setmainfont[ItalicFont={brilli.ttf}, BoldFont={brillb.ttf}, Numbers=Lining]{brill.ttf}
\usefonttheme{serif}
\usepackage{microtype}
\usepackage{multicol}

\newcommand{\Subitem}[1]{{\setlength\itemindent{12pt} \item[-] #1}}
\newcommand{\Subsubitem}[1]{{\setlength\itemindent{24pt} \item[○] #1}}
\newcommand{\Subsubsubitem}[1]{{\setlength\itemindent{36pt} \item[-] #1}}
\newcommand{\Subsubsubsubitem}[1]{{\setlength\itemindent{48pt} \item[○] #1}}
\DeclareMathOperator*{\argmax}{arg\,max}

\title{Meeting 31: Vedic hydronyms}
\subtitle{The Rivers of the R̥g-vedá}
\author{Nikhil Surya Dwibhashyam}
\date{13 November 2022}

\begin{document}

\frame{\titlepage}

\begin{frame} \frametitle{Why a Véda discussion group?}
\begin{itemize}
	\item What is Vedic literature?
	\Subitem Sáṁhitā-s
	\Subitem Brā́hmaṇa-s
	\Subitem Others?
	\item Véda-s vs.~Vedā́nta, Upaniṣád-s, Purāṇá-s
	\Subitem Later monism (ádvāita) vs.~earlier dualism
	\Subitem Moral--political subtexts
	\item Modern (religious, scholarly) fixation upon later literature
\end{itemize}
\end{frame}

\begin{frame} \frametitle{What is there to discuss?}
\begin{itemize}
	\item History \& linguistics
	\item Metaphysics of religion
	\item Exegesis (e.g.~western vs.~Eastern narratives)
	\item Moral principles
	\item Nuances of Ṡrāutá ritual
	\item How we can apply Vedic principles to our lives
\end{itemize}
\end{frame}

\begin{frame} \frametitle{Our plans}
\begin{itemize}
	\item Weekly ``general audience'' meetings on Sundays at 1000 EST
	\Subitem Followed by in-depth literature meeting at 1130 EST
	\item Different topic/aspect/angle discussed every week
	\item Format:
	\Subitem Præsentation
	\Subitem Free-form discussion
	\item Eventually something more
	\Subitem Children's books
	\Subitem Poëtry collective
	\Subitem Documentation
	\Subitem Actual rituals
\end{itemize}
\end{frame}

\begin{frame} \frametitle{WhatsApp group}
\begin{center}
	\href{https://chat.whatsapp.com/IXCQEkhfrcwHI7CNY8Fgat}{https://chat.whatsapp.com/IXCQEkhfrcwHI7CNY8Fgat}
\end{center}
\end{frame}

\begin{frame} \frametitle{Let's introduce ourselves!}
\begin{itemize}
	\item Briefly!
	\item Name
	\item School (if student)
	\item How you found this group
	\item Background/interest in Hinduism \& Vedic literature
\end{itemize}
\end{frame}

\begin{frame}[label=questions] \frametitle{Some questions to consider}
\begin{itemize}
	\item Which rivers are mentioned in the Véda?
	\item What evidence do we have for correspondence with real-life rivers?
	\item Might names have been reassigned to certain rivers over time?
	\item What about the concept of mythical rivers that don't physically exist?
\end{itemize}
\end{frame}

\begin{frame} \frametitle{Words for rivers}
\begin{itemize}
	\item \textit{Nadī́} < $\surd$\textit{nad}
	\Subitem cf.~Gk.~\textit{Néda}, Gm. \textit{Nette}
	\Subitem 61 times in R̥V 
	\item \textit{Síndʰu} < $\surd$\textit{sidʰ}?
	\Subitem 173 times in R̥V 
	\item Other words for waters: áp (generic), aváni, kulyā́, dʰāutí, sarít, sáras
\end{itemize}
\end{frame}

\begin{frame} \frametitle{The geography of the R̥g-vedá}
\begin{center}
	Aṣṭāú vá akʰyat • kakúbʰaḥ Pr̥tʰivyā́ḥ

	trī́ dʰánua • yójanā saptá síndʰūn.

	Hiraṇyā-'kṣáḥ Savitā́ • Devá ā́ 'gād

	dádʰad rátnā • dāşúṣe vā́riāṇi.

	\vspace{12pt}

	He illuminates the Earth's eight points,

	the three desert regions and the seven rivers.

	The golden-eyed God Savitŕ̥ has come,

	giving choice treasures to the worshiper.
\end{center}
\end{frame}

\begin{frame} \frametitle{The river-hymn: 10.75}
\begin{itemize}
	\item Síndʰu
	\Subitem Áşvā ná, 7
	\item Gáṅgā
	\Subitem ʰápax!
	\item Yamúnā
	\item Sárasvatī
	\item Ṡutudrī́
	\item Páruṣṇī
	\item Ásiknī
	\item Ārjīkī́ya
\end{itemize}
\end{frame}

\begin{frame} \frametitle{The river-hymn: 10.75 (contd.)}
\begin{itemize}
	\item Tr̥ṣṭā́mā
	\item Rasā́
	\item Susártu
	\item Ṡvetyā́
	\item Kúbʰā
	\item Gomatī́
	\item Krúmu
	\item Mehatnū́
	\item Sīlámāvatī
	\item Ū́rṇāvatī
\end{itemize}
\end{frame}

\begin{frame} \frametitle{Rivers mentioned in the R̥g-vedá}
\begin{itemize}
	\item Many others!
	\Subitem Suṣómā \&c.
	\item Pay attention to names themselves \& their meaning
	\item Stratigraphy
	\item Different texts
	\item Historical events
\end{itemize}
\end{frame}

\begin{frame} \frametitle{The elusive Sárasvatī}
\begin{itemize}
	\item Real-life identity
	\Subitem Mythical identity
	\item Avestan correspondence
	\item Ghaggar-Hakra
	\item Drying up?
\end{itemize}
\end{frame}

\againframe{questions}

\end{document}
