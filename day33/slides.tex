\documentclass[pdf]{beamer}
\mode<presentation>{}
\usetheme{Rochester}
\usecolortheme{whale}
\beamertemplatenavigationsymbolsempty

\usepackage[ngerman]{babel}
\usepackage{fontspec}
\setmainfont[ItalicFont={brilli.ttf}, BoldFont={brillb.ttf}, Numbers=Lining]{brill.ttf}
\usefonttheme{serif}
\usepackage{microtype}
\usepackage{multicol}

\newcommand{\Subitem}[1]{{\setlength\itemindent{12pt} \item[-] #1}}
\newcommand{\Subsubitem}[1]{{\setlength\itemindent{24pt} \item[○] #1}}
\newcommand{\Subsubsubitem}[1]{{\setlength\itemindent{36pt} \item[-] #1}}
\newcommand{\Subsubsubsubitem}[1]{{\setlength\itemindent{48pt} \item[○] #1}}
\DeclareMathOperator*{\argmax}{arg\,max}

\title{Meeting 33: Music in the Véda}
\subtitle{Instruments, chants, \& songs}
\author{Nikhil Surya Dwibhashyam}
\date{27 November 2022}

\begin{document}

\frame{\titlepage}

\begin{frame} \frametitle{Why a Véda discussion group?}
\begin{itemize}
	\item What is Vedic literature?
	\Subitem Sáṁhitā-s
	\Subitem Brā́hmaṇa-s
	\Subitem Others?
	\item Véda-s vs.~Vedā́nta, Upaniṣád-s, Purāṇá-s
	\Subitem Later monism (ádvāita) vs.~earlier dualism
	\Subitem Moral--political subtexts
	\item Modern (religious, scholarly) fixation upon later literature
\end{itemize}
\end{frame}

\begin{frame} \frametitle{What is there to discuss?}
\begin{itemize}
	\item History \& linguistics
	\item Metaphysics of religion
	\item Exegesis (e.g.~western vs.~Eastern narratives)
	\item Moral principles
	\item Nuances of Ṡrāutá ritual
	\item How we can apply Vedic principles to our lives
\end{itemize}
\end{frame}

\begin{frame} \frametitle{Our plans}
\begin{itemize}
	\item Weekly ``general audience'' meetings on Sundays at 1000 EST
	\Subitem Followed by in-depth literature meeting at 1130 EST
	\item Different topic/aspect/angle discussed every week
	\item Format:
	\Subitem Præsentation
	\Subitem Free-form discussion
	\item Eventually something more
	\Subitem Children's books
	\Subitem Poëtry collective
	\Subitem Documentation
	\Subitem Actual rituals
\end{itemize}
\end{frame}

\begin{frame} \frametitle{WhatsApp group}
\begin{center}
	\href{https://chat.whatsapp.com/IXCQEkhfrcwHI7CNY8Fgat}{https://chat.whatsapp.com/IXCQEkhfrcwHI7CNY8Fgat}
\end{center}
\end{frame}

\begin{frame} \frametitle{Let's introduce ourselves!}
\begin{itemize}
	\item Briefly!
	\item Name
	\item School (if student)
	\item How you found this group
	\item Background/interest in Hinduism \& Vedic literature
\end{itemize}
\end{frame}

\begin{frame}[label=questions] \frametitle{Some questions to consider}
\begin{itemize}
	\item What was the role of music in Vedic society?
	\item How was music used in the ritual?
	\item How did music evolve across the Vedic period?
	\item What basis might later Indian musical traditions have in the Véda?
\end{itemize}
\end{frame}

\begin{frame} \frametitle{The whole ritual is music}
\begin{itemize}
	\item Cʰándas
	\item Yáti
	\item Derivation from earlier meter
	\item Éka-şruti?
\end{itemize}
\end{frame}

\begin{frame} \frametitle{Music as art}
\begin{itemize}
	\item Kauṣītaka-brā́hmaṇa 22.5: Ṡílpa
	\Subitem Nr̥tya
	\Subitem Gītá
	\Subitem Vāditá
	\item Vā́ṇīcī? R̥V 5.75.4
\end{itemize}
\end{frame}

\begin{frame} \frametitle{Instruments in war}
\begin{itemize}
	\item Dundubʰí: R̥V 1.28.5, 6.47.29, 6.47.31
	\item Bákura: R̥V 1.117.21
	\item Bā́kura: R̥V 9.1.8
	\item Also bekurā
\end{itemize}
\end{frame}

\begin{frame} \frametitle{Instruments in ritual}
\begin{itemize}
	\item ``Jaritŕ̥''
	\item Gárgara, godʰā́, píṅgā: R̥V 8.69.9
	\item KS 34.5 and accompanying ṠS: detailed description of Mahāvratá
	\Subitem Avagʰaṭarikā
	\Subitem Ghāṭa-karkarī
	\Subitem Vāṇá/vāṇī́
	\Subitem Alābu-vīṇā
	\Subitem Kāṇḍa-vīṇā
	\Subitem Piccʰorā
\end{itemize}
\end{frame}

\begin{frame} \frametitle{Civil music}
\begin{itemize}
	\item Āgʰāṭí: R̥V 10.146.2, AV 4.37.4
	\item Musician as a profession, e.g. talavá (VS 30.20), āḍambarāgʰātá (VS 30.19), jyākārá (VS 30.7)
	\item Nītʰa: musical mode
	\item Dʰarma-şāstrá references
\end{itemize}
\end{frame}

\begin{frame} \frametitle{Sāma-vedá}
\begin{itemize}
	\item Demonstration
	\item Notation
	\item Basis for later music?
\end{itemize}
\end{frame}

\againframe{questions}

\end{document}
