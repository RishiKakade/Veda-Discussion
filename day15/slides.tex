\documentclass[pdf]{beamer}
\mode<presentation>{}
\usetheme{Rochester}
\usecolortheme{whale}
\beamertemplatenavigationsymbolsempty

\usepackage{fontspec}
\setmainfont[ItalicFont={brilli.ttf}, BoldFont={brillb.ttf}, BoldItalicFont={brillbi.ttf}]{brill.ttf}
\usefonttheme{serif}
\usepackage{microtype}
\usepackage{multicol}

\newcommand{\Subitem}[1]{{\setlength\itemindent{12pt} \item[-] #1}}
\newcommand{\Subsubitem}[1]{{\setlength\itemindent{24pt} \item[○] #1}}
\newcommand{\Subsubsubitem}[1]{{\setlength\itemindent{36pt} \item[-] #1}}
\newcommand{\Subsubsubsubitem}[1]{{\setlength\itemindent{48pt} \item[○] #1}}

\title{Meeting 15: Māyā́ in the Vedic religion}
\subtitle{Wisdom? magic? illusion?}
\author{Nikhil Surya Dwibhashyam}
\date{5 June 2022}

\begin{document}

\frame{\titlepage}

\begin{frame}[label=questions] \frametitle{Some questions to consider}
\begin{itemize}
	\item What was māyā́, originally?
	\Subitem What were its different aspects within the Vedic religion?
	\item How did māyā́ evolve in the post-Vedic period?
	\Subitem In the Purāṇá-s?
	\item How did Ṡramaṇá \& Vedā́nta change the connotations of māyā́?
	\item How are non-Hindu conceptions of māyā́ to be distinguished?
\end{itemize}
\end{frame}

\begin{frame} \frametitle{Māyā́ in the R̥g-vedá’s lexicon}
\begin{block} {māyā́ f.~(MW on R̥V)}
\begin{itemize}
	\item Art, wisdom, extraordinary or supernatural power
	\item Illusion, unreality, deception, fraud, trick, sorcery, witchcraft, magic
	\item An unreal or illusory image, phantom, apparition
\end{itemize}
\end{block}
\pause
\begin{block} {māyā́ f.~(Graßmann)}
\begin{itemize}
	\item $\surd$\textit{man}, vgl.~\textit{mātí}
	\item Übermenschliche Weisheit oder List, göttliche Kunst oder Zauberkunst, Zauberbild, Trugbild
	\item Weise oder listige Anschläge (pl.)
	\item Zauberei von bösen, mit Dämonen in Verbindung gedachten Menschen
\end{itemize}
\end{block}
\end{frame}

\begin{frame} \frametitle{Māyā́ in the R̥g-vedá’s lexicon}
\begin{itemize}
	\item Mentioned 39 times in R̥V
	\item Good
	\Subitem Mitrásya Váruṇasya
	\Subitem Indrásya
	\Subitem Ásurasya
	\Subitem Of the Sun \& Moon
	\Subitem Of Tváṣṭr̥
	\item Bad
	\Subitem Dā́sasya
	\Subitem Ádevasya Vr̥trásya
	\Subitem Dānavásya
\end{itemize}
\end{frame}

\begin{frame} \frametitle{The māyā-bʰedá hymn (10.177)}
\begin{itemize}
	\item Very varied interpretations
	\item Sun? bird? more spiritual?
	\item What is māyā́ here? illusion?
\end{itemize}
\end{frame}

\begin{frame} \frametitle{Māyā́ in the Atʰarva-vedá}
\begin{itemize}
	\item Generally same as in R̥V
	\item But more emphasis on sorcery aspect given context
	\item Also, importantly: AV 8.10.22
	\Subitem Māyā́ supposedly ``personified''
	\Subitem Really title for a woman
\end{itemize}
\end{frame}

\begin{frame} \frametitle{Māyā́ \& morality}
\begin{itemize}
	\item Is māyā́ deceit?
	\item Is māyā́ a bad thing?
	\item Similarity to Abrahamic religions?
\end{itemize}
\end{frame}

\begin{frame} \frametitle{Māyā́ outside the Véda}
\begin{itemize}
	\item Emphasis on illusion develops
	\item Emphasis on deceit still present
	\item Emphasis on power \& magic?
\end{itemize}
\end{frame}

\againframe{questions}

\begin{frame} \frametitle{Interesting read}
\begin{itemize}
	\item Hager, B. ``Die Entwicklung des Māyā-Begriffes im Indo-Arischen.'' \textbf{1983}.
\end{itemize}
\end{frame}

\end{document}
