\documentclass[pdf]{beamer}
\mode<presentation>{}
\usetheme{Rochester}
\usecolortheme{whale}
\beamertemplatenavigationsymbolsempty

\usepackage{fontspec}
\setmainfont[ItalicFont={brilli.ttf}, BoldFont={brillb.ttf}]{brill.ttf}
\usefonttheme{serif}
\usepackage{microtype}
\usepackage{multicol}

\newcommand{\Subitem}[1]{{\setlength\itemindent{12pt} \item[-] #1}}
\newcommand{\Subsubitem}[1]{{\setlength\itemindent{24pt} \item[○] #1}}
\newcommand{\Subsubsubitem}[1]{{\setlength\itemindent{36pt} \item[-] #1}}
\newcommand{\Subsubsubsubitem}[1]{{\setlength\itemindent{48pt} \item[○] #1}}

\title{Meeting 4: From Agástya to Vyáşva}
\subtitle{The bards of the R̥g-vedá \& their families}
\author{Nikhil Surya Dwibhashyam, Rohan Pandey}
\date{13 March 2022}

\begin{document}

\frame{\titlepage}

\begin{frame} \frametitle{WhatsApp group}
\begin{center}
	DM me your phone number in the chat if you want to be added.
\end{center}
\end{frame}

\begin{frame}[label=questions] \frametitle{Some questions to consider}
\begin{itemize}
	\item How can the bardic assignments of the R̥g-vedá be helpful in historical reconstruction?
	\item To what extent do today's Brāhmaṇá-s continue the Vedic bardic tradition?
	\item What can be done to revive the tradition of poëtic composition?
	\item We'll return to these at the end!
\end{itemize}
\end{frame}

\begin{frame} \frametitle{What is a Brahmin?}
\begin{itemize}
	\item Skt.~\textit{Brāhmaṇá}: vŕ̥ddʰi derivative of \textit{bráhman}
	\Subitem not to be confused with \textit{brā́hmaṇa}
	\item What is \textit{bráhman}?
	\Subitem Nominalization of $\surd$\textit{br̥h}
	\Subsubitem Dual meaning: be strong, increase
	\Subsubitem Aryan expansionism?
	\Subitem Triple meaning: prayer, pray-er (priest), pray-ee (god)
	\Subitem Not related to Lat.~\textit{flāmen}
\end{itemize}
\end{frame}

\begin{frame} \frametitle{What is a Brahmin?}
\begin{itemize}
	\item Brāhmaṇá: collective term for priests (r̥tvíj-s)
	\Subitem Bráhman
	\Subitem Hótr̥
	\Subitem Adʰvaryú
	\Subitem Ud-gātŕ̥
	\Subitem Each with (3?) assistants
	\item Profession-based \emph{and} lineage-based
	\Subitem Patrilineal (and usually also maternal) descent
	\Subitem Patriarchal
	\item Central to the religious--social order of the Ā́rya-s
\end{itemize}
\end{frame}

\begin{frame} \frametitle{What did the Brāhmaṇá-s do?}
\begin{itemize}
	\item Undergo extensive training (under specialty)
	\item Bless/consecrate the ruling order
	\Subitem e.g.~coronation, horse sacrifice, battles
	\Subitem More on this later!
	\item Officiate the Ṡrāutá rituals
	\item Officiate many home rituals (birth, marriage, \&c)
	\item Memorize \& pass down ancient hymns
	\item But also, more importantly to this meeting: write poëtry!
	\Subitem Hymns of prayer
	\Subitem Hymns of praise
	\Subsubitem To men
	\Subsubitem To gods
\end{itemize}
\end{frame}

\begin{frame} \frametitle{The lineage system of the bardic families}
\begin{itemize}
	\item Gotrá (e.g.~Vā́lakʰilya)
	\Subitem Pravará (e.g.~Agʰamarṣaṇa–Kuşiká)
	\Subsubitem Vaṁşá (e.g.~Ārāma-Drāviḍa-Brāhmaṇá)
	\Subsubsubitem Kúla (e.g.~Dvibʰā́ṣya)
	\Subsubsubsubitem Víş
	\item Dví-gotra?
\end{itemize}
\end{frame}

\begin{frame} \frametitle{What was the relationship between bards \& hymns?}
\begin{itemize}
	\item Revealed or composed?
	\Subitem Straightforward answer: composed.
	\Subitem ``Newest hymn''
	\Subitem Over several centuries
	\Subsubitem {\emph{Vedic Metre in Its Historical Development}}
	\Subsubsubitem Arnold, E.~V. \textbf{1905}
	\item Innovation \& inventiveness praised
	\Subitem But not \emph{too} innovative!
	\item Metrical elegance
	\Subitem Integral to Ṡrāutá ritual
\end{itemize}
\end{frame}

\begin{frame} \frametitle{Who were the bards of the R̥g-vedá?}
\begin{itemize}
	\item Some famous ones:
	\Subitem Áṅgiras
	\Subitem Vişvā́mitra
	\Subitem Vásiṣṭʰa
	\Subitem Bʰŕ̥gu
	\item How do we know?
	\Subitem Both inside \& outside the R̥g-vedá
	\Subsubitem e.g.~\textit{táv’ ét tát satyám Aṅgiraḥ} (1.1.6)
\end{itemize}
\end{frame}

\begin{frame} \frametitle{Different bards, different themes}
\begin{itemize}
	\item Different styles
	\Subitem Meter, vocabulary, grammar
	\Subitem Diachronic!
	\item Different gods?
	\Subitem e.g.~Vā́lakʰilya \& his focus on Índra
	\Subitem Possible precusor to kula-devá system?
	\item Different heroes \& kings
	\Subitem Purūrávas, Kútsa, Dasyavevŕ̥ka, \&c
	\Subitem Ten Kings
	\Subitem {\textbf{Intimate connection between Brāhmaṇá-s and Kṣatríya-s}}
	\item Surprisingly little scholarship
	\Subitem ``Implicit research''
\end{itemize}
\end{frame}

\begin{frame} \frametitle{Indian bards today}
\begin{itemize}
	\item Often non-Brāhmaṇá
	\Subitem e.g.~Bʰāt-s of Rajasthan
	\Subitem Nonetheless, attempts at Brahminization
	\item Similar relationship in some ways
	\Subitem Songs of praise to men and gods
	\item Bʰajaná composers
\end{itemize}
\end{frame}

\againframe{questions}

\end{document}
