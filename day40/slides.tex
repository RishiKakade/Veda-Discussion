\documentclass[pdf]{beamer}
\mode<presentation>{}
\usetheme{Rochester}
\usecolortheme{whale}
\beamertemplatenavigationsymbolsempty

\usepackage[ngerman]{babel}
\usepackage{fontspec}
\setmainfont[ItalicFont={brilli.ttf}, BoldFont={brillb.ttf}, Numbers=Lining]{brill.ttf}
\usefonttheme{serif}
\usepackage{microtype}
\usepackage{multicol}

\newcommand{\Subitem}[1]{{\setlength\itemindent{12pt} \item[-] #1}}
\newcommand{\Subsubitem}[1]{{\setlength\itemindent{24pt} \item[○] #1}}
\newcommand{\Subsubsubitem}[1]{{\setlength\itemindent{36pt} \item[-] #1}}
\newcommand{\Subsubsubsubitem}[1]{{\setlength\itemindent{48pt} \item[○] #1}}
\DeclareMathOperator*{\argmax}{arg\,max}

\title{Meeting 40: Wives, mothers, rivers}
\subtitle{The Goddesses of the Véda}
\author{Nikhil Surya Dwibhashyam}
\date{22 January 2023}

\begin{document}

\frame{\titlepage}

\begin{frame} \frametitle{Why a Véda discussion group?}
\begin{itemize}
	\item What is Vedic literature?
	\Subitem Sáṁhitā-s
	\Subitem Brā́hmaṇa-s
	\Subsubitem ĀŞS 24.1.31: Mántra--Brā́hmaṇayor Veda-nāma-dʰeyám
	\Subitem Others? What is Şrúti?
	\item Véda-s vs.~Vedāntá, Upaniṣád-s, Itihāsá-s, Purāṇá-s
	\Subitem Later philosophies like monism (ádvāita)
	\Subitem Moral--political subtexts
	\item Modern (religious, scholarly) fixation upon later literature
\end{itemize}
\end{frame}

\begin{frame} \frametitle{What is there to discuss?}
\begin{itemize}
	\item History \& linguistics
	\item Metaphysics of religion
	\item Exegesis (e.g.~western vs.~Eastern narratives)
	\item Moral principles
	\item Nuances of Şrāutá ritual
	\item How we can apply Vedic principles to our lives
\end{itemize}
\end{frame}

\begin{frame} \frametitle{Our plans}
\begin{itemize}
	\item Weekly ``general audience'' meetings on Sundays at 1000 EST
	\Subitem Followed by in-depth literature meeting at 1130 EST
	\item Different topic/aspect/angle discussed every week
	\item Format:
	\Subitem Præsentation
	\Subitem Free-form discussion
	\item Eventually something more
	\Subitem Children's books
	\Subitem Poëtry collective
	\Subitem Documentation
	\Subitem Actual rituals
\end{itemize}
\end{frame}

\begin{frame} \frametitle{WhatsApp group}
\begin{center}
	\href{https://chat.whatsapp.com/IXCQEkhfrcwHI7CNY8Fgat}{https://chat.whatsapp.com/IXCQEkhfrcwHI7CNY8Fgat}
\end{center}
\end{frame}

\begin{frame} \frametitle{Let's introduce ourselves!}
\begin{itemize}
	\item Briefly!
	\item Name
	\item School (if student)
	\item How you found this group
	\item Background/interest in Hinduism \& Vedic literature
\end{itemize}
\end{frame}

\begin{frame} \frametitle{Contents}
\begin{itemize}
	\item What are the Goddesses?
	\item Uṣás
	\Subitem Rā́trī
	\item Vā́c
	\Subitem ``Devī́-sūkta''
	\item Indrāṇī́
	\item Varuṇānī́, Agnā́yī; Rudrāṇī́; Sū́ryā
	\item Áditi, Díti
	\item Pr̥thivī́
	\item Apsarás-s, Áp-s, rivers
	\item In the sacrifice
\end{itemize}
\end{frame}

\begin{frame} \frametitle{Goddess}
\begin{itemize}
	\item \textit{Devī́}: feminine of \textit{Devá}
	\Subitem (from \textit{√div})
	\item Metaphysically and grammatically: feminine counterparts
	\item Primarily: \textit{Devá-patnī}; \textit{Devā́nām pátnī}
	\item Some repræsenting primary natural phænomena
\end{itemize}
\end{frame}

\begin{frame} \frametitle{Uṣás}
\begin{itemize}
	\item The Dawn: primary natural phænomenon
	\item Highly honored! >20 hymns, >300 times in R̥V
	\item Praised for beauty
	\item Awakens all
	\item Breath of everything
	\item Wife of Sū́rya
\end{itemize}
\end{frame}

\begin{frame} \frametitle{Rātrī́}
\begin{itemize}
	\item Praised much less (R̥V 10.127)
	\item Still: this praise is theologically important
	\Subitem ``Evil'' connotations?
\end{itemize}
\end{frame}

\begin{frame} \frametitle{Wives of the Gods}
\begin{itemize}
	\item Rarely important mythological role
	\item Feminine counterparts
	\item Indrāṇī́!
	\item Varuṇānī́, Agnā́yī; Rudrāṇī́; Sū́ryā
\end{itemize}
\end{frame}

\begin{frame} \frametitle{Mothers of the Gods}
\begin{itemize}
	\item Áditi, Díti
	\item Pr̥thivī́
\end{itemize}
\end{frame}

\begin{frame} \frametitle{In the sacrifice}
\begin{itemize}
	\item Usually offered oblations alongside Theri husbands
	\item But lesser!
	\item e.g. Paşu-bandʰa (ĀŞS 7.27.10): portion of tail
\end{itemize}
\end{frame}

\end{document}
