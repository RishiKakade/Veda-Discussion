\documentclass[pdf]{beamer}
\mode<presentation>{}
\usetheme{Rochester}
\usecolortheme{whale}
\beamertemplatenavigationsymbolsempty

\usepackage[ngerman]{babel}
\usepackage{fontspec}
\setmainfont[ItalicFont={brilli.ttf}, BoldFont={brillb.ttf}, BoldItalicFont={brillbi.ttf}]{brill.ttf}
\usefonttheme{serif}
\usepackage{microtype}
\usepackage{multicol}

\newcommand{\Subitem}[1]{{\setlength\itemindent{12pt} \item[-] #1}}
\newcommand{\Subsubitem}[1]{{\setlength\itemindent{24pt} \item[○] #1}}
\newcommand{\Subsubsubitem}[1]{{\setlength\itemindent{36pt} \item[-] #1}}
\newcommand{\Subsubsubsubitem}[1]{{\setlength\itemindent{48pt} \item[○] #1}}

\title{Meeting 22: Svargá, Nā́raka, \& more}
\subtitle{The Vedic afterlife}
\author{Nikhil Surya Dwibhashyam}
\date{28 August 2022}

\begin{document}

\frame{\titlepage}

\begin{frame} \frametitle{Meeting agenda}
\begin{itemize}
	\item (Re)introduction to discussion group
	\item Today's topic præsentation
	\item Free discussion
\end{itemize}
\end{frame}

\begin{frame} \frametitle{Why a Véda discussion group?}
\begin{itemize}
	\item What is Vedic literature?
	\Subitem Sáṁhitā-s
	\Subitem Brā́hmaṇa-s
	\Subitem Others?
	\item Véda-s vs.~Vedā́nta,  Upaniṣád-s, Purāṇá-s
	\Subitem Later monism (ádvāita) vs.~earlier dualism
	\item Modern (religious, scholarly) fixation upon later literature
\end{itemize}
\end{frame}

\begin{frame} \frametitle{What is there to discuss?}
\begin{itemize}
	\item History \& linguistics
	\item Metaphysics of religion
	\item Exegesis (e.g.~countering western narratives)
	\item Moral principles
	\item Nuances of Ṡrāutá ritual
\end{itemize}
\end{frame}

\begin{frame} \frametitle{Our plans}
\begin{itemize}
	\item Weekly meetings on Sundays at 1200 EDT
	\Subitem Followed by more in-depth literature meeting at 1330 EDT
	\item Different topic/aspect/angle discussed in depth every week
	\item Format:
	\Subitem Præsentation
	\Subitem Free-form discussion
	\item Eventually something more
\end{itemize}
\end{frame}

\begin{frame} \frametitle{WhatsApp group}
\begin{center}
	\href{https://chat.whatsapp.com/IXCQEkhfrcwHI7CNY8Fgat}{https://chat.whatsapp.com/IXCQEkhfrcwHI7CNY8Fgat}
\end{center}
\end{frame}

\begin{frame} \frametitle{Let's introduce ourselves!}
\begin{itemize}
	\item Name
	\item School (if student)
	\item How you found this group
	\item Briefly: background/interest in Hinduism \& Vedic literature
\end{itemize}
\end{frame}

\begin{frame}[label=questions] \frametitle{Some questions to consider}
\begin{itemize}
	\item Did the afterlife conceptions significantly change within the Vedic period?
	\Subitem Is the Vedic literature internally consistent on the afterlife?
	\item What factors decided one's fate in the afterlife?
	\item What is \emph{our} fate in the Vedic afterlife, as modern people?
\end{itemize}
\end{frame}

\begin{frame} \frametitle{Immortality?}
\begin{itemize}
	\item \textit{Amŕ̥ta} usually used of the Gods
	\item But several times used for men too, e.g.~\textit{ápāma sómam • amŕ̥tā abʰūma} (R̥V 8.48.3)
	\item Must refer to an ever-lasting \emph{afterlife}
	\item Implication: annihilation is possible
\end{itemize}
\end{frame}

\begin{frame} \frametitle{The funeral rites}
\begin{itemize}
	\item More detailed discussion in future meeting
	\item Cremation \emph{and} burial (not incompatible), e.g.~R̥V 10.18.11
	\item Just as in sacrifice, spirit carried to Heavens by Agní
	\item Probably the charred remains were interred, alongside possessions.
	\item Animal sacrifice alongside?
	\item Saha-gamana?
\end{itemize}
\end{frame}

\begin{frame} \frametitle{Fates after death}
\begin{itemize}
	\item Four-fold:
	\Subitem Annihilation
	\Subitem Hell
	\Subitem Heaven
	\Subitem Reïncarnation
	\item Decided by piousness \& virtue on Earth, e.g.~how much sacrificed
	\Subitem \& by actions of your descendants
\end{itemize}
\end{frame}

\begin{frame} \frametitle{Annihilation}
\begin{itemize}
	\item i.e.~no afterlife at all for some people
	\item Little discussion, but very likely
	\item Sometimes Nírr̥ti interpreted so: ``destruction''
	\Subitem cf.~use even aside from afterlife, e.g. R̥V 10.10.11
	\Subitem But this is unlikely: specific fates are described, e.g. wolves (R̥V 10.95.14)
\end{itemize}
\end{frame}

\begin{frame} \frametitle{Hell}
\begin{itemize}
	\item Nā́raka (cf.~later Naraka)
	\item Vivid descriptions even in Vedic literature: especially R̥V 7.104 (for dæmons)
	\Subitem Thus: dæmons \& men alike can go there
	\Subitem Boiling in water
	\Subitem Given to serpent (áhi)
	\Subitem Darkness, depths/caverns
	\Subitem Chewing hair \& swimming in blood (AV 5.19.3)
	\Subitem \&c.~\&c.
\end{itemize}
\end{frame}

\begin{frame} \frametitle{Svargá}
\begin{itemize}
	\item Maps fairly well onto Heaven
	\Subitem {\emph{Literally} in the Heavens: great souls shine as stars (TS 5.4.1.3, ṠB 6.5.4.8)}
	\Subitem Also cf.~etymology
	\item Vivid ``hedonistic'' description in AV 4.34
	\Subitem {Sexual pleasure: \textit{n' ā́iṣāṁ şişnám • prá dahati jātávedāḥ}}
	\Subsubitem Later reflexes of this in MBʰ \&c.
	\Subitem Lakes of butter, honey, wine; filled with lotus-blossoms
	\item {Other names, e.g.~\textit{ásunīti} (R̥V 10.16.2), \textit{pitr̥-loká}}
\end{itemize}
\end{frame}

\begin{frame} \frametitle{Reïncarnation}
\begin{itemize}
	\item Clearly præsent in the Vedic religion, but little discussion
	\item Rebirth for evil-doërs: strange creature (ṠB 3.1.2.12)
	\item Reïncarnation as descendants?
	\item Consistency with Svargá
\end{itemize}
\end{frame}

\begin{frame} \frametitle{Some references}
\begin{itemize}
	\item Bodewitz, H. ``Life after death in the Ṛgveda Saṁhitā.'' \textit{Gonda Indol.~Stud.}, \textbf{2019}.
	\item Chakraborty, C. ``The racial history of India.'' \textbf{1944}.
	\item Shusan, G. ``Afterlife conceptions in the Vedas.'' \textit{Religion Compass.}, \textbf{2011}.
\end{itemize}
\end{frame}


\againframe{questions}

\end{document}
