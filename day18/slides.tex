\documentclass[pdf]{beamer}
\mode<presentation>{}
\usetheme{Rochester}
\usecolortheme{whale}
\beamertemplatenavigationsymbolsempty

\usepackage[ngerman]{babel}
\usepackage{fontspec}
\setmainfont[ItalicFont={brilli.ttf}, BoldFont={brillb.ttf}, BoldItalicFont={brillbi.ttf}]{brill.ttf}
\usefonttheme{serif}
\usepackage{microtype}
\usepackage{multicol}

\newcommand{\Subitem}[1]{{\setlength\itemindent{12pt} \item[-] #1}}
\newcommand{\Subsubitem}[1]{{\setlength\itemindent{24pt} \item[○] #1}}
\newcommand{\Subsubsubitem}[1]{{\setlength\itemindent{36pt} \item[-] #1}}
\newcommand{\Subsubsubsubitem}[1]{{\setlength\itemindent{48pt} \item[○] #1}}

\title{Meeting 18: The jána-s \& their organization}
\subtitle{Nationhood in the Vedic religion}
\author{Nikhil Surya Dwibhashyam}
\date{3 July 2022}

\begin{document}

\frame{\titlepage}

\begin{frame} \frametitle{Meeting agenda}
\begin{itemize}
	\item (Re)introduction to discussion group
	\item Today's topic præsentation
	\item Free discussion
\end{itemize}
\end{frame}

\begin{frame} \frametitle{Why a Véda discussion group?}
\begin{itemize}
	\item What is Vedic literature?
	\Subitem Sáṁhitā-s
	\Subitem Brā́hmaṇa-s
	\Subitem Others?
	\item Véda-s vs.~Vedā́nta \& Upaniṣád-s
	\Subitem Later monism (ádvāita) vs.~earlier dualism (dvāitá)
	\item Modern (religious, scholarly) fixation upon later literature
\end{itemize}
\end{frame}

\begin{frame} \frametitle{What is there to discuss?}
\begin{itemize}
	\item History \& linguistics
	\item Metaphysics of religion
	\item Exegesis (e.g.~countering western narratives)
	\item Moral principles
	\item Nuances of Ṡrāutá ritual
\end{itemize}
\end{frame}

\begin{frame} \frametitle{Our plans}
\begin{itemize}
	\item Weekly meetings on Sundays at 1200 EDT
	\Subitem Followed by more in-depth literature meeting at 1330 EDT
	\item Different topic/aspect/angle discussed in depth every week
	\item Format:
	\Subitem Præsentation
	\Subitem Free-form discussion
	\item Eventually something more
\end{itemize}
\end{frame}

\begin{frame} \frametitle{WhatsApp group}
\begin{center}
	\href{https://chat.whatsapp.com/IXCQEkhfrcwHI7CNY8Fgat}{https://chat.whatsapp.com/IXCQEkhfrcwHI7CNY8Fgat}
\end{center}
\end{frame}

\begin{frame} \frametitle{Let's introduce ourselves!}
\begin{itemize}
	\item Name
	\item School (if student)
	\item How you found this group
	\item Briefly: background/interest in Hinduism \& Vedic literature
\end{itemize}
\end{frame}

\begin{frame}[label=questions] \frametitle{Some questions to consider}
\begin{itemize}
	\item How did the ancient Ā́rya tribes come to be?
	\item How were the ancient Ā́rya tribes socially organized?
	\item What historical developments took place in the Ā́rya tribes across time?
	\item To what extent did the ancient Ā́rya-s constitute a nation?
\end{itemize}
\end{frame}

\begin{frame} \frametitle{The nation}
\begin{itemize}
	\item Community of people with shared features
	\Subitem Language (intelligibility)
	\Subitem Genetics (kinship or phenotype)
	\Subitem History (shared struggle)
	\Subitem Beliefs (religion)
	\Subitem Conjunction of these often necessary in the past
	\item Perennialism vs.~modernization theory
	\item Shared interests
	\item Organization-agnostic
\end{itemize}
\end{frame}

\begin{frame} \frametitle{The Ā́rya designation}
\begin{itemize}
	\item A group of Sanskrit-speaking tribes originating in NW India
	\Subitem Allied in war, but also had internecine battles
	\Subitem Comparable to flower wars?
	\item Distinct from western academic term “Aryan”
	\Subitem Which includes Iranic peoples Ā́rya-s would consider Dásyu 
	\item \& from Gobineau's even broader use for all IE peoples
	\item Genetic aspect (obviously)
	\Subitem Patrilineal descent
	\Subitem Not the same as “steppe DNA”, R1a, ANI, \&c.
	\Subsubitem {\textit{Ā́rya} is a binary.}
	\Subitem Adoption?
	\item United most of all by religion: \textit{aryá}
\end{itemize}
\end{frame}

\begin{frame} \frametitle{Caste: the common fate of IE peoples}
\begin{itemize}
	\item Invading ethnic group conquers another
	\Subitem Forms élite group in new region
	\Subitem Over generations: \textit{prajā́ dakṣiṇā́ 'bʰi-jáyantīr yanti} (MS 4.7.9)
	\Subitem {\textit{Ā́rya várṇa} (R̥V 3.34.9) vs.~\textit{Dāsá várṇa} (R̥V 2.12.4)}
	\Subitem White vs.~black: not skin color
	\item Herrenmoral: the rule of the free man
	\Subitem Subjugation of conquered group \& eradication of their faiths
	\Subsubitem {\textit{pañcā-şátaṁ vadʰū́nām} (R̥V 8.19.36), \textit{şatáṁ Dāsā́n} (R̥V 8.56.3)}
	\Subitem Less complete in India due to population density
	\item Conquered group integrated into society of conquerors
	\Subitem Ṡūdrá-s: slaves/servants, later peasants
\end{itemize}
\end{frame}

\begin{frame} \frametitle{Caste: the common fate of IE peoples}
\begin{itemize}
	\item Élite aristocracy \emph{within} conquering group
	\Subitem Dumézil's “trifunctional hypothesis”, várṇa (estate/color)
	\Subitem Brāhmaṇá-s (white): priests responsible (solely) for religion
	\Subitem Kṣatríya-s (red): warriors responsible (solely) for lordship
	\Subitem Some amount of exchange between them, cf. Vişvā́mitra
	\Subitem Endogamous lineages from very beginning
	\item Víş (black): commoners among conquering group
	\Subitem Associated with “productive functions”, e.g. gopā́-s
	\Subitem cf. GobʰGS 4.7.5 for colors
	\item Priests \& warriors dominated commoners.
	\Subitem {\textit{Vāíşya máyi, yát te paró níhitaṁ, tád ā́' hara} (ṠB 1.3.2.15)}
	\Subitem Similar to (pagan \& Christian) European nobility
\end{itemize}
\end{frame}

\begin{frame} \frametitle{The jána-s}
\begin{itemize}
	\item Spread across region from Upárişyena to Víndʰya-s
	\Subitem South India considered alien to Ā́rya-s until centuries later
	\item From Dʰarma-şāstrá-s onwards: Āryāvartá
	\item Individually conquered neighboring territories \& kine
	\Subitem From non-Ā́rya-s through Vrā́ta raids
	\Subitem From each other through war (gáv-iṣṭi)
	\item Five great tribes (R̥V 8.10.5): Ánu, Turváşa, Druhyú, Pūrú, Yádu
	\item United under Aryamán (guardian of Ā́rya customs)
	\Subitem Bound by oaths to one another under Mitrá–Váruṇa:
	\Subsubitem An Ā́rya may attack another, but not break an oath.
\end{itemize}
\end{frame}

\begin{frame} \frametitle{The jána-s}
\begin{itemize}
	\item Élite spoke various dialects of Vedic Sanskrit
	\Subitem \& commoners a language not too divergent
	\Subitem As opposed to the mr̥dʰrá-vāc-s (R̥V 1.172.2)
	\item Fascinating dialectal differences: see Witzel
	\Subitem Gods speak differently from men, but still Sanskrit (MS 1.5.12)
	\Subitem Northerners had best speech
	\Subitem Woman spoke more clearly, used “popular” forms (\textit{kuru})
	\Subitem Also diachronic differences: see Arnold
	\item Jána-s often corresponded with şā́kʰā-s
\end{itemize}
\end{frame}

\begin{frame} \frametitle{Internecine conflict}
\begin{itemize}
	\item Ten Kings' Battle (R̥V 7.18)
	\Subitem Probably \emph{not} related to the Mahā-bʰārata
	\Subitem Five tribes with Paktʰá-s, Bʰalānás-s, Álina-s, Viṣāṇín-s
	\Subitem Against Tŕ̥tsu-s, Bʰaratá-s
	\Subitem Gods intervened, decided victory of Tŕ̥tsu-s
	\Subitem Vişvā́mitra served Sudā́s
	\item Plenty of other conflicts between fully Ā́rya tribes mentioned
	\item Decidedly less violent than any other conflicts!
\end{itemize}
\end{frame}

\begin{frame} \frametitle{Social organization within the jána}
\begin{itemize}
	\item Rule by a king: consistent throughout Véda
	\item Expansionism through aşva-medʰá (e.g. R̥V 1.162)
	\item Consecration at rāja-sū́ya (e.g. R̥V 10.173)
	\item Division of labor between men \& women
	\Subitem \& of course by caste
	\item Progenitor of four classic stages of life
\end{itemize}
\end{frame}

\begin{frame} \frametitle{Consolidation}
\begin{itemize}
	\item Towards end of Vedic period
	\item Kuru–Pañcāla \& Kosala--Videha were two hegemons
	\Subitem But many outlying tribes still
	\item Paved way for pan-subcontinental empires of later India
	\item Consolidation of ritual as well
	\Subitem No more “newest hymn”!
	\item Vedic religion became much more centralized
	\Subitem Functioned as more formalized state religion like Romans'
	\item “Hindu synthesis” well under way
\end{itemize}
\end{frame}

\begin{frame} \frametitle{Ā́rya nationhood in later India}
\begin{itemize}
	\item Traditional Vedic rituals still used to legitimize rule
	\Subitem Aşva-medʰá on Gupta coinage, for example
	\Subitem But only 2 aşva-medʰá-s in past millennium!
	\item Vedic dynasties \& names of jánas frequent in literature
	\Subitem Tribal names (likely real) basis for Mahā-bʰārata lineages
	\item Political movements like “dvijá identitarianism”
\end{itemize}
\end{frame}

\begin{frame} \frametitle{Some references}
\begin{itemize}
	\item Rau, W. ``Staat und Gesellschaft im alten Indien'' \textbf{1957}.
	\item Witzel, M. ``Tracing Vedic dialects.'' \textbf{1989}.
	\item Witzel, M. ``Early Sanskritization.'' \textit{Elec. Journ. Vedic Stud.}, \textbf{1995}.
	\item Witzel, M. ``Autochthonous Aryans?'' \textit{Elec. Journ. Vedic Stud.}, \textbf{2001}.
\end{itemize}
\end{frame}

\againframe{questions}

\end{document}
