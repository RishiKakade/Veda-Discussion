\documentclass[pdf]{beamer}
\mode<presentation>{}
\usetheme{Rochester}
\usecolortheme{whale}
\beamertemplatenavigationsymbolsempty

\usepackage[ngerman]{babel}
\usepackage{fontspec}
\setmainfont[ItalicFont={brilli.ttf}, BoldFont={brillb.ttf}, BoldItalicFont={brillbi.ttf}]{brill.ttf}
\usefonttheme{serif}
\usepackage{microtype}
\usepackage{multicol}

\newcommand{\Subitem}[1]{{\setlength\itemindent{12pt} \item[-] #1}}
\newcommand{\Subsubitem}[1]{{\setlength\itemindent{24pt} \item[○] #1}}
\newcommand{\Subsubsubitem}[1]{{\setlength\itemindent{36pt} \item[-] #1}}
\newcommand{\Subsubsubsubitem}[1]{{\setlength\itemindent{48pt} \item[○] #1}}

\title{Meeting 19: A tale of two faiths}
\subtitle{Comparing the Véda and the Avesta}
\author{Nikhil Surya Dwibhashyam}
\date{24 July 2022}

\begin{document}

\frame{\titlepage}

\begin{frame} \frametitle{Meeting agenda}
\begin{itemize}
	\item (Re)introduction to discussion group
	\item Today's topic præsentation
	\item Free discussion
\end{itemize}
\end{frame}

\begin{frame} \frametitle{Why a Véda discussion group?}
\begin{itemize}
	\item What is Vedic literature?
	\Subitem Sáṁhitā-s
	\Subitem Brā́hmaṇa-s
	\Subitem Others?
	\item Véda-s vs.~Vedā́nta \& Upaniṣád-s
	\Subitem Later monism (ádvāita) vs.~earlier dualism (dvāitá)
	\item Modern (religious, scholarly) fixation upon later literature
\end{itemize}
\end{frame}

\begin{frame} \frametitle{What is there to discuss?}
\begin{itemize}
	\item History \& linguistics
	\item Metaphysics of religion
	\item Exegesis (e.g.~countering western narratives)
	\item Moral principles
	\item Nuances of Ṡrāutá ritual
\end{itemize}
\end{frame}

\begin{frame} \frametitle{Our plans}
\begin{itemize}
	\item Weekly meetings on Sundays at 1200 EDT
	\Subitem Followed by more in-depth literature meeting at 1330 EDT
	\item Different topic/aspect/angle discussed in depth every week
	\item Format:
	\Subitem Præsentation
	\Subitem Free-form discussion
	\item Eventually something more
\end{itemize}
\end{frame}

\begin{frame} \frametitle{WhatsApp group}
\begin{center}
	\href{https://chat.whatsapp.com/IXCQEkhfrcwHI7CNY8Fgat}{https://chat.whatsapp.com/IXCQEkhfrcwHI7CNY8Fgat}
\end{center}
\end{frame}

\begin{frame} \frametitle{Let's introduce ourselves!}
\begin{itemize}
	\item Name
	\item School (if student)
	\item How you found this group
	\item Briefly: background/interest in Hinduism \& Vedic literature
\end{itemize}
\end{frame}

\begin{frame}[label=questions] \frametitle{Some questions to consider}
\begin{itemize}
	\item What are some characteristics of the Avestan languages?
	\item What features do the Vedic and Avestan religions have in common?
	\item What interactions may have existed between the two?
	\item To what extent can the Avestan religion help us reconstruct the Vedic?
\end{itemize}
\end{frame}

\begin{frame} \frametitle{What is the Vedic religion?}
\begin{itemize}
	\item National faith of the ethno-religious Ā́rya-s
	\item Texts: Sáṁhitā-s (hymns), Brā́hmaṇa-s (exegesis)
	\item Language: accented Vedic Sanskrit (Indo-Aryan)
	\item Centered around sacrifice (yajñá) to the Gods (Devá-s) mediated by fire (Agní)
	\item Significantly different from later Hinduism
\end{itemize}
\end{frame}

\begin{frame} \frametitle{What is the Avestan religion?}
\begin{itemize}
	\item National faith of the ethno-religious ``Arya-s''
	\item Texts: Yasna, Vendidad (Vīdaēvōdāta)
	\Subitem Former divided into various sections (e.g.~Gāθā-s)
	\item Languages: unaccented Old and Young Avestan (Iranic)
	\item Contains sacrifice (yasna) to god mediated by fire
	\item Significantly different from later Zoroastrianism
\end{itemize}
\end{frame}

\begin{frame} \frametitle{Geography}
\begin{itemize}
	\item Somewhat overlapping regions initially
	\item Sárasvatī vs.~Haraxᵛaitī?
	\Subitem Probably not: cf.~etym.
	\item Aryanəm Vaējah
	\item When was the split?
	\item Much closer affinity to Vedic than other Iranic branches have
	\item Dāsá, Dásyu = Dāha, Dahyu
\end{itemize}
\end{frame}

\begin{frame} \frametitle{Language}
\begin{center}
	\textit{aibigairiiā daiϑē vīspā humatācā hūxtācā huuarštācā}

	abi-garyā datʰe vīspā hu-matā cā hū-’kʰtā cā hv-arştā cā

	Mutually intelligible? No.
\end{center}
\end{frame}

\begin{frame} \frametitle{Theology}
\begin{itemize}
	\item Monotheism vs.~polytheism?
	\Subitem Not really: there are many yazata-s
	\item Good vs.~evil
	\item Ásura, Devá = Ahura, Daēwa
	\Subitem Specific Gods too: Nāŋhaiθya, Indra, Vərəθraγna, Apąm Napāt, Miθra
	\item Other cognates:
	\Subitem Āθrauuan, baga, \&c.
\end{itemize}
\end{frame}

\begin{frame} \frametitle{Cosmology}
\begin{itemize}
	\item Aša (vs.~Druj) \& R̥tá: cosmic order
	\Subitem Binary good vs.~evil common to both
	\Subitem Light vs.~dark
	\item Creation myths
	\Subitem Púruṣa, gav-aēvō.dātā
	\Subitem The first mortal: Yamá (Yima)
	\item Others?
\end{itemize}
\end{frame}

\begin{frame} \frametitle{Cultural practices}
\begin{itemize}
	\item Death: cremation \& burial vs.~daxma (root in Vendidad)
	\item Incest: taboo (Yamá \& Yamī́) vs.~valued (Xᵛaētwadaθa)
	\item Cattle-slaughter: taboo vs.~allowed
	\item Role of priesthood
\end{itemize}
\end{frame}

\begin{frame} \frametitle{Utility in Vedic reconstruction?}
\begin{itemize}
	\item Morality: not too useful (see above)
	\item Sóma: identity, præparation, ritual
	\item Form of ritual in general
	\item Original identity of certain tribes
	\item Ontology?
\end{itemize}
\end{frame}

\begin{frame} \frametitle{Some references}
\begin{itemize}
	\item Witzel, M. ``The home of the Aryans.'' \textbf{1998}.
	\item Witzel, M. ``Early Sanskritization.'' \textit{Elec. Journ. Vedic Stud.}, \textbf{1995}.
	\item Witzel, M. ``Autochthonous Aryans?'' \textit{Elec. Journ. Vedic Stud.}, \textbf{2001}.
\end{itemize}
\end{frame}

\againframe{questions}

\end{document}
