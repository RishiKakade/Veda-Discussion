\documentclass[pdf]{beamer}
\mode<presentation>{}
\usetheme{Rochester}
\usecolortheme{whale}
\beamertemplatenavigationsymbolsempty

\usepackage[ngerman]{babel}
\usepackage{fontspec}
\setmainfont[ItalicFont={brilli.ttf}, BoldFont={brillb.ttf}, BoldItalicFont={brillbi.ttf}]{brill.ttf}
\usefonttheme{serif}
\usepackage{microtype}
\usepackage{multicol}

\newcommand{\Subitem}[1]{{\setlength\itemindent{12pt} \item[-] #1}}
\newcommand{\Subsubitem}[1]{{\setlength\itemindent{24pt} \item[○] #1}}
\newcommand{\Subsubsubitem}[1]{{\setlength\itemindent{36pt} \item[-] #1}}
\newcommand{\Subsubsubsubitem}[1]{{\setlength\itemindent{48pt} \item[○] #1}}

\title{Meeting 20: From the Kŕ̥ttikā-s to the Bʰáraṇī-s}
\subtitle{Exploring Vedic asterisms and astrolomy}
\author{Nikhil Surya Dwibhashyam}
\date{7 August 2022}

\begin{document}

\frame{\titlepage}

\begin{frame} \frametitle{Meeting agenda}
\begin{itemize}
	\item (Re)introduction to discussion group
	\item Today's topic præsentation
	\item Free discussion
\end{itemize}
\end{frame}

\begin{frame} \frametitle{Why a Véda discussion group?}
\begin{itemize}
	\item What is Vedic literature?
	\Subitem Sáṁhitā-s
	\Subitem Brā́hmaṇa-s
	\Subitem Others?
	\item Véda-s vs.~Vedā́nta \& Upaniṣád-s
	\Subitem Later monism (ádvāita) vs.~earlier dualism (dvāitá)
	\item Modern (religious, scholarly) fixation upon later literature
\end{itemize}
\end{frame}

\begin{frame} \frametitle{What is there to discuss?}
\begin{itemize}
	\item History \& linguistics
	\item Metaphysics of religion
	\item Exegesis (e.g.~countering western narratives)
	\item Moral principles
	\item Nuances of Ṡrāutá ritual
\end{itemize}
\end{frame}

\begin{frame} \frametitle{Our plans}
\begin{itemize}
	\item Weekly meetings on Sundays at 1200 EDT
	\Subitem Followed by more in-depth literature meeting at 1330 EDT
	\item Different topic/aspect/angle discussed in depth every week
	\item Format:
	\Subitem Præsentation
	\Subitem Free-form discussion
	\item Eventually something more
\end{itemize}
\end{frame}

\begin{frame} \frametitle{WhatsApp group}
\begin{center}
	\href{https://chat.whatsapp.com/IXCQEkhfrcwHI7CNY8Fgat}{https://chat.whatsapp.com/IXCQEkhfrcwHI7CNY8Fgat}
\end{center}
\end{frame}

\begin{frame} \frametitle{Let's introduce ourselves!}
\begin{itemize}
	\item Name
	\item School (if student)
	\item How you found this group
	\item Briefly: background/interest in Hinduism \& Vedic literature
\end{itemize}
\end{frame}

\begin{frame}[label=questions] \frametitle{Some questions to consider}
\begin{itemize}
	\item What is the significance of the nákṣatra-s to the Vedic religion?
	\Subitem in later Hinduism?
	\item What sectarian trends might have existed in the relevance of the nákṣatra-s?
	\item What \emph{are} the nákṣatra-s metaphysically? How were they formed?
	\item To what extent are Vedic astronomy \& astrology compatible with modern science?
	\Subitem Literally? metaphorically?
	\item What elements of the Vedic astrological tradition can be found in modern Hindu astrology?
\end{itemize}
\end{frame}

\begin{frame} \frametitle{Vedic astronomy \& cosmology}
\begin{itemize}
	\item \textit{Pari-maṇḍalá u vā́ ayáṁ lokáḥ} (ṠB 7.1.1.37): ``This world indeed is round.''
	\item \textit{Sa vā eṣa na kadā can' āstam eti n' od eti.  Taṁ yad astam etī 'ti manyante, 'hna eva tad antam itvā 'tʰ' ātmānaṁ vi pary asyate.  Rātrīm ev' āvastāt kurute 'haḥ parastāt} (AB 3.44.6--7): ``Never indeed does He set nor rise. When it is thought that He sets, having arrived at the day's end, then He turns Himself over.  He makes night below \& day above.''
	\item Possible heliocentrism (prob.~not): \textit{Tád Ādityá imā́ṁ lokā́n sū́tre sám ā́ vayate} (ṠB 8.7.3.10): ``Then the Sun strings himself to these worlds on a thread.''
	\Subitem Geocentrism as a metaphysical--moral position
	\item The Earth has 7 regions (R̥V 1.22.16).
\end{itemize}
\end{frame}

\begin{frame} \frametitle{Vedic astronomy \& cosmology (contd.)}
\begin{itemize}
	\item \textit{Ádʰi tvíṣīr • adʰita Sū́riasya} (R̥V 9.71.9): ``He (Sóma) hath assumed Sū́rya's brilliance.''
	\item Sacrifice necessary for Sun-rise (ṠB 2.3.1.5)?  Other interpretations are possible.
	\item The Sun is frequently called a star (\textit{nákṣatra vişvá-deva}).
	\item ``Eclipse-causing dæmon'' Súarbʰānu (R̥V 5.40) doesn't refer to eclipse
	\item No notion of cyclical time or later Hindu ideas like the Manv-ántara-s
	\item One view: creation ``as-is'' by Gods (Young Earth Creationism)
	\Subitem Alternatively, more metaphorical views (Old Earth)
\end{itemize}
\end{frame}

\begin{frame} \frametitle{Vedic astrology vs.~``Vedic astrology''}
\begin{itemize}
	\item Modern Indian astrology is a Greek import (cf.~Yavaneşvara) integrated with local (incl.~Vedic) traditions.
	\item Horoscopes, personality traits, Zodiac, week-days, Mangalik, \&c. totally absent from Vedic period \& in some cases even præ-mediæval period
	\item Vedic astrology used to appoint times for sacrificial (\& later other) ritual
	\item That is: fundamentally \textbf{not prædictive}
	\item Nákṣatra-s were limit of Vedic astronomy/astrology
	\Subitem Little if any discussion of planets, for example
\end{itemize}
\end{frame}

\begin{frame} \frametitle{What is a nákṣatra?}
\begin{itemize}
	\item Prob.~$\surd$\textit{nakṣ} ``attain'' + \textit{-tra} ``means''
	\Subitem Aufrecht: \textit{nákta} ``night'' + \textit{-tra} ``protection''
	\Subitem ṠB 2.1.2.19 (``kabbalistic'' etymology genre): \textit{ná} + \textit{kṣatrá} ``power''
	\item ``Asterism'': constellation but unofficial
	\item Also used synonymously with \textit{stŕ̥}
	\item ``Lunar mansions'': division of 27--28 days, segments of the Moon's ecliptic
\end{itemize}
\end{frame}

\begin{frame} \frametitle{Stŕ̥ in the R̥g-vedá}
\begin{itemize}
	\item {\textit{Pipéşa nā́kaṁ • stŕ̥bʰir dámūnāḥ} (1.68.10): ``The householder (Agní) adorned the sky with stars.''}
	\item {\textit{Aruṣásya • duhitárā ví-rūpe: / stŕ̥bʰir anyā́ • pipişé Sū́rav anyā́} (6.49.3): ``The Red's daughters are unlike: one adorned with stars, the other the Sun's.''}
	\Subitem i.e.~Day \& Night
\end{itemize}
\end{frame}

\begin{frame} \frametitle{Nákṣatra(-s) in the R̥g-vedá}
\begin{itemize}
	\item {\textit{Ápa tyé tāyávo yatʰā / nákṣatrā yanti aktúbʰiḥ / Sū́rāya vişvá-cakṣase.} (1.50.2): ``Like thieves the constellations go away with their light before the all-seeing Sun.''}
	\item {\textit{Ṡr̥ṇótu naḥ • Pr̥tʰivī́ Dyāúr ut' ā́paḥ / Sū́ryo nákṣatrāir • urú antárikṣam} (3.54.19): ``May they hear us: the Earth, the Heavens, the waters, the Sun, the stars, the wide space.''}
	\item {\textit{Dr̥ḷhó nákṣatra • utá vişvá-devaḥ / Bʰū́mim ā́ atān • Diā́ṁ dʰāsín' āyóḥ} (6.67.6): ``Fixed, the star of all gods (Sū́rya) / filled the Earth \& the Heavens with the food of life.''}
	\item {\textit{Nákṣatrebʰiḥ • Pitáro Dyā́m apiṁşan} (10.68.11): ``The Fathers decorated the Heavens with stars.''}
	\item {\textit{Átʰo nákṣatrāṇām eṣā́m / upástʰe Sóma ā́hitaḥ} (10.85.2): ``Thus among the stars in this place Sóma (the Moon) is set.''}
	\item {\textit{Nákṣatram pratnám • áminac cariṣṇú} (10.88.13): ``The ancient star (Agní) eternally wandering''}
\end{itemize}
\end{frame}

\begin{frame} \frametitle{Nákṣatra-s in the Atʰarva-vedá (19.7.1)}
\begin{center}
\begin{multicols}{2}
\small{
	Citrā́ṇi sākáṁ • Diví rocanā́ni

	sarīsr̥pā́ṇi • bʰúvane javā́ni

	aṣṭā-viṁşáṁ • su-matím ichámānaḥ

	áhāni gīrbʰíḥ • saparyāmi nā́kam.

	\vspace{12pt}

	\textit{Shining bright together in the Heavens,}

	\textit{gliding through the world in haste:}

	\textit{I, seeking the favor of those twenty-eight,}

	\textit{worship the days \& the sky with songs.}
}
\end{multicols}
\end{center}
\end{frame}

\begin{frame} \frametitle{Nákṣatra-s in the Atʰarva-vedá (19.7.2)}
\begin{center}
	Su-hávam me • Kŕ̥ttikā Róhiṇī ca

	ástu bʰadráṁ • Mr̥gáşiraḥ şám Ārdrā́

	Púnarvasū • sūnŕ̥tā cā́ru Púṣyaḥ

	bʰānúr āşleṣā́ • áyanaṁ Magʰā́ me.

	\vspace{12pt}

	\textit{May the \textbf{Kŕ̥ttikā-s} (1) \& \textbf{Róhiṇī} (2) be well-invoked for me,}

	\textit{be \textbf{Mr̥gáşiras} (3) blessed, \textbf{Ārdrā́} (4) healthy,}

	\textit{be the \textbf{Púnarvasu-s} (5) pleasant, \textbf{Púṣya} (6) agreeable,}

	\textit{be the \textbf{Āşleṣā́-s} (7) bright, the \textbf{Magʰā́-s} (8) progress for me.}
\end{center}
\end{frame}

\begin{frame} \frametitle{Nákṣatra-s in the Atʰarva-vedá (19.7.3)}
\begin{center}
	Púṇyaṁ Pū́rvā • Pʰálgunyāu c’ ā́tra Hástaḥ

	Citrā́ şivā́ • Svātí sukʰó may astu

	Rā́dʰo Vişā́kʰe • su-háv’ Ānurādʰā́

	Jyéṣṭʰā su-nákṣatram • áriṣṭa Mū́lam.

	\vspace{12pt}

	\textit{May the \textbf{Former Pʰálgunī-s} (9) [\& the \textbf{Latter} (10)] \& \textbf{Hásta} (11) be virtue,}

	\textit{be \textbf{Citrā́} (12) holy \& \textbf{Svātí} (13) easy for me,}

	\textit{be the \textbf{Vişā́kʰā-s} (14) bounty \& \textbf{Ānurādʰā́} (15) well-invoked,}

	\textit{be \textbf{Jyéṣṭʰā} (16) well-starred \& \textbf{Mū́la} (17) uninjured.}
\end{center}
\end{frame}

\begin{frame} \frametitle{Nákṣatra-s in the Atʰarva-vedá (19.7.4)}
\begin{center}
	Ánnam Pū́rvā • rāsatām may Aṣādʰā́ḥ

	ū́rjaṁ Devī́ • Úttarā ā́ vahātu

	Abʰijín me • rāsatāṁ púṇyam evá

	Ṡróṇaḥ Ṡráviṣṭʰāḥ • kurvatāṁ su-puṣṭím.

	\vspace{12pt}

	\textit{May the \textbf{Former Aṣādʰā́-s} (18) grant me food,}

	\textit{the \textbf{Latter} (19) Goddesses bring me strength,}

	\textit{\textbf{Abʰijít} (20) grant me virtue indeed,}

	\textit{\textbf{Ṡróṇa} (21) \& the \textbf{Ṡráviṣṭʰā-s} (22) make good prosperity.}
\end{center}
\end{frame}

\begin{frame} \frametitle{Nákṣatra-s in the Atʰarva-vedá (19.7.5)}
\begin{center}
	Ā́ me mahác • Cʰatábʰiṣag várīyaḥ

	ā́ me dvayā́ • Próṣṭʰapadā su-şárma

	ā́ Revátī • c’ Āşvayújāu bʰágam me

	ā́ me rayím •  Bʰáraṇya ā́ vahantu.

	\vspace{12pt}

	\textit{May \textbf{Ṡatábʰiṣaj} (23) bring me great freedom,}

	\textit{the two [Former and Latter] \textbf{Próṣṭʰapadā-s} (24, 25) good refuge to me,}

	\textit{\textbf{Revátī} (26) and the \textbf{Aşvayúj-s} (27) fortune to me,}

	\textit{the \textbf{Bʰáraṇī-s} (28) treasure to me.}
\end{center}
\end{frame}

\begin{frame} \frametitle{Nákṣatra-s in the Tāittirīya-sáṁhitā}
\begin{itemize}
	\item Some differences in enumeration (4.4.10): Tiṣyà for Púṣya, Svātī́ for Svātí, Róhiṇī (repeated) for Jyéṣṭʰā, Vicŕ̥t-s for Mū́la, Apabʰáraṇī-s for Bʰáraṇī-s
	\item Each nákṣatra assigned to a God, e.g.~Kŕ̥ttikā-s to Agní, Róhiṇī to Prajā́pati
	\item Nákṣatra-s appointed for certain rites: e.g.~consecration on Citrā́ (TS 7.4.8), Púnarvasu-s for rekindling (TS 1.5.1)
	\item Nákṣatra-s themselves hailed in many rituals; also nákṣatra-bricks and -offerings
	\item \textit{Sukŕ̥tāṁ vā́ etā́ni jyótī̃ṣi yán nákṣatrāṇi} (TS 5.4.1.3): ``The lights of the well-doërs, these are the stars.''
	\item Enumeration in Tāittirīya-brā́hmaṇa aligns with TS
	\Subitem Much more detailed descriptions, esp.~TB 3.1.1–2
	\Subitem See also TB 1.1.2, 1.5.1.
\end{itemize}
\end{frame}

\begin{frame} \frametitle{Nákṣatra-s in the Ṡata-patʰa-brā́hmaṇa}
\begin{itemize}
	\item Similar situation to KYV in ritual hailing
	\item No enumeration of nákṣatra-s like in KYV, nor assignment to Gods
	\item Fewer ritual præscriptions, but still some:
	\Subitem Funerals should be performed in a ``singular'' nákṣatra (ṠB 8.8.1.3).
	\item The Sun took away (\textit{ā́ $\surd$dā}) the power of the stars, and can serve as a nákṣatra (ṠB 2.1.2.18–19).
	\item \textit{Nákṣatrāṇi vāí jánayo, yé hí jánāḥ puṇya-kŕ̥taḥ Svargáṁ lokáṁ yánti} (ṠB 6.5.4.8): ``The stars are people, who acting virtuously go to Svargá.''
	\item \textit{Nákṣatrāṇi vāí sárveṣāṃ Devā́nām āyátanam} (ṠB 14.3.2.12): ``The stars are the home of all the Gods.''
\end{itemize}
\end{frame}

\begin{frame} \frametitle{Vedic vestiges today}
\begin{itemize}
	\item Original Vedic nákṣatra-s still used today by astrologers
	\item Certain rituals must be performed at ordained, auspicious times.
	\item How divergent from Vedic præscriptions?
\end{itemize}
\end{frame}

\begin{frame} \frametitle{Interesting blog posts}
\begin{itemize}
	\item mAnasa-taraMgiNI. ``nakShatra homa as per the kAThaka yajurvedins.'' \textbf{2007}.
	\item mAnasa-taraMgiNI. ``A note on the asterisms forming the nakṣatra-s.'' \textbf{2017}.
\end{itemize}
\end{frame}

\againframe{questions}

\end{document}
