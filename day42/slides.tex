\documentclass[pdf]{beamer}
\mode<presentation>{}
\usetheme{Rochester}
\usecolortheme{whale}
\beamertemplatenavigationsymbolsempty

\usepackage[ngerman]{babel}
\usepackage{fontspec}
\setmainfont[ItalicFont={brilli.ttf}, BoldFont={brillb.ttf}, Numbers=Lining]{brill.ttf}
\usefonttheme{serif}
\usepackage{microtype}
\usepackage{multicol}

\newcommand{\Subitem}[1]{{\setlength\itemindent{12pt} \item[-] #1}}
\newcommand{\Subsubitem}[1]{{\setlength\itemindent{24pt} \item[○] #1}}
\newcommand{\Subsubsubitem}[1]{{\setlength\itemindent{36pt} \item[-] #1}}
\newcommand{\Subsubsubsubitem}[1]{{\setlength\itemindent{48pt} \item[○] #1}}
\DeclareMathOperator*{\argmax}{arg\,max}

\title{Meeting 42: Booty \& trade}
\subtitle{The Vedic œconomy}
\author{Nikhil Surya Dwibhashyam}
\date{12 February 2023}

\begin{document}

\frame{\titlepage}

\begin{frame} \frametitle{Why a Véda discussion group?}
\begin{itemize}
	\item What is Vedic literature?
	\Subitem Sáṁhitā-s
	\Subitem Brā́hmaṇa-s
	\Subsubitem ĀŞS 24.1.31: Mántra--Brā́hmaṇayor Veda-nāma-dʰeyám
	\Subitem Others? What is Şrúti?
\end{itemize}
\end{frame}

\begin{frame} \frametitle{Why a Véda discussion group?}
\begin{itemize}
	\item Véda-s vs.~Upaniṣád-s, Itihāsá-s, Purāṇá-s, Vedāntá
	\Subitem AV 15.6.11--12?
	\Subitem Later philosophies like monism (ádvāita)
	\Subitem Moral--political subtexts
	\item Modern (religious, scholarly) fixation upon later literature
\end{itemize}
\end{frame}

\begin{frame} \frametitle{What is there to discuss?}
\begin{itemize}
	\item History \& linguistics
	\item Metaphysics of Vedic religion
	\item Exegetical narratives
	\Subitem Western vs.~Eastern
	\Subitem Scholarly ``vs.''~religious
	\item Moral principles
	\item Nuances of Şrāutá ritual
	\item Applying Vedic principles to our lives
\end{itemize}
\end{frame}

\begin{frame} \frametitle{Our plans}
\begin{itemize}
	\item Weekly ``general audience'' meetings on Sundays at 1000 EST
	\Subitem Followed by in-depth literature meeting at 1130 EST
	\item Different topic/aspect/angle discussed every week
	\item Format:
	\Subitem Præsentation
	\Subitem Free-form discussion
	\item Eventually something more
	\Subitem Poëtry composition
	\Subitem Children's books
	\Subitem Documentation
	\Subitem Actual rituals
\end{itemize}
\end{frame}

\begin{frame} \frametitle{WhatsApp group \& Discord server}
\begin{center}
	\href{https://chat.whatsapp.com/IXCQEkhfrcwHI7CNY8Fgat}{https://chat.whatsapp.com/IXCQEkhfrcwHI7CNY8Fgat}
	\href{https://discord.gg/F5ExYJKrMk}{https://discord.gg/F5ExYJKrMk}
\end{center}
\end{frame}

\begin{frame} \frametitle{Let's introduce ourselves!}
\begin{itemize}
	\item Briefly!
	\item Name
	\item School (if student) or where you're based
	\item How you found this group
	\item Background/interest in Hinduism \& Vedic literature
\end{itemize}
\end{frame}

\begin{frame}[label=questions] \frametitle{Discussion questions}
\begin{itemize}
	\item How did the Vedic Ā́rya-s earn their living?
	\Subitem How did it change across time?
	\item How advanced were Vedic œconomics? What financial systems were in place?
	\item 
\end{itemize}
\end{frame}

\begin{frame} \frametitle{Contents}
\begin{itemize}
	\item On nomadism
	\item Means of living
	\Subitem By caste
	\item Agriculture
	\item Trade
	\Subitem Currency
	\Subitem Barter
	\Subitem Sóma
	\item Plunder
	\Subitem The assembly
	\Subitem Distribution
	\item Property rights
	\Subitem Slavery
	\Subitem Inheritance
\end{itemize}
\end{frame}

\begin{frame} \frametitle{The meaning of life}
\begin{itemize}
	\item The Classical puruṣ'-ārtha-s
	\Subitem Dhárma
	\Subitem Ártha
	\Subitem Kā́ma
	\Subitem Mokṣa (non-Vedic)
\end{itemize}
\end{frame}

\begin{frame} \frametitle{The meaning of life (contd.)}
\begin{itemize}
	\item The Vedic goals of life (my own interpretation)
	\Subitem Ā́yus
	\Subitem Şrī́
	\Subitem Şrávas
	\Subitem Svargá (most importantly)
	\item ``Materialism''
	\Subitem The nature of the yajñá
\end{itemize}
\end{frame}

\begin{frame} \frametitle{Œconomic virtues in the Véda}
\begin{itemize}
	\item Wealth
	\Subitem Be rich!
	\Subitem {\textit{Gatá-şrī}}
	\Subitem By virtuous means
	\Subsubitem ``Greedy behavior?''
	\item Charity
	\Subitem Be generous!
	\Subitem R̥V 10.117.4
\end{itemize}
\end{frame}

\begin{frame} \frametitle{Nomadic life}
\begin{itemize}
	\item Continued proliferation of Ā́rya
	\Subitem ``Āryāvartá''
	\Subitem Over generations: \textit{prajā́ dakṣiṇā́ 'bʰi-jáyantīr yanti} (MS 4.7.9)
	\item Nomadism can be a misnomer
	\Subitem But: the Vrā́tya-s
	\item Better: pastoralism
	\Subitem Rural vs.~urban life
	\Subitem The grā́ma
\end{itemize}
\end{frame}

\begin{frame} \frametitle{Means of living}
\begin{itemize}
	\item Priestly
	\item Plunder
	\item Largest class: sæcular
	\Subitem Thriving art
	\Subitem Artisans
	\Subitem Herders
	\Subitem \&c. \&c.
\end{itemize}
\end{frame}

\begin{frame} \frametitle{Agriculture}
\begin{itemize}
	\item The calendar
	\item Sacrifices centered around agriculture
	\Subitem Şunāsīrīya
	\item 
\end{itemize}
\end{frame}

\begin{frame} \frametitle{Food}
\begin{itemize}
	\item Grains: yáva (barley), dʰānā́, sáktu, odaná
	\item Beans: mā́ṣa, mudgá
	\item Fruit (pʰála): melon (urvāruká)
	\item Aquatic plants: ávakā
	\item \& others, e.g.~sóma, alcohol (súrā, e.g.~R̥V 1.116.7)
	\item But chiefly, animal products
	\Subitem Honey (mádʰu)
	\Subitem Dairy: milk (kṣīrá, dugdʰá, páyas), curd (dadʰán), butter (sarpís), ghee (gʰr̥tá), sour cream (mástu), mixture (āmíkṣā)
	\Subitem Meat (mā́s, e.g.~R̥V 1.161.10): buffalo (mahiṣá, e.g.~R̥V 5.29.8), goat (ajá, e.g.~R̥V 1.162.4), sheep (meṣá, e.g.~R̥V 10.91.14); sacrificially horses (e.g.~R̥V 1.162.9) \& kine (e.g.~R̥V 10.86.14)
\end{itemize}
\end{frame}

\begin{frame} \frametitle{Currency}
\begin{itemize}
	\item Niṣka
	\item Kraya
	\item Hiraṇya
\end{itemize}
\end{frame}

\begin{frame} \frametitle{The dákṣiṇā}
\begin{itemize}
	\item Priestly fees
	\Subitem Specific
	\item Often quite exorbitant
	\item But often restrictions (see Dhárma \& later literature)
	\Subitem Most often kine and gold
\end{itemize}
\end{frame}

\begin{frame} \frametitle{The sóma-sale}
\begin{itemize}
	\item Ritualized incorporation of Şūdrá
	\item Originally Gandharva-s
\end{itemize}
\end{frame}

\begin{frame} \frametitle{Plunder}
\begin{itemize}
	\item Of enemies
	\item Of Ā́rya-s
	\item Within a tribe?
	\Subitem Theft
	\Subsubitem Bad nails
\end{itemize}
\end{frame}

\begin{frame} \frametitle{The vidátha}
\begin{itemize}
	\item Also: sabhā́
	\item Distribution of plunder by the king
	\item What is known about it?
\end{itemize}
\end{frame}

\begin{frame} \frametitle{Property rights}
\begin{itemize}
	\item Stirner or Nietzsche, or whoever else you præfer
	\Subitem Based on force
	\item {\textit{Vāíşya máyi, yát te paró níhitaṁ, tád ā́' hara} (ŞB 1.3.2.15)}
	\item Women?
	\item Şūdrá-s: anyasya preṣya, kāmotthāpya, yathākāmavadhya (AB)
	\Subitem Bahu-paşu, pādāvanejya (PB)
\end{itemize}
\end{frame}

\begin{frame} \frametitle{References}
\begin{itemize}
	\item MacDonell, A.~A., Keith, A.~B., \textbf{1912}. \textit{The Vedic index of names \& subjects.}
\end{itemize}
\end{frame}

\againframe{questions}

\end{document}
