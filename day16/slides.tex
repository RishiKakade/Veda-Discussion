\documentclass[pdf]{beamer}
\mode<presentation>{}
\usetheme{Rochester}
\usecolortheme{whale}
\beamertemplatenavigationsymbolsempty

\usepackage[ngerman]{babel}
\usepackage{fontspec}
\setmainfont[ItalicFont={brilli.ttf}, BoldFont={brillb.ttf}, BoldItalicFont={brillbi.ttf}]{brill.ttf}
\usefonttheme{serif}
\usepackage{microtype}
\usepackage{multicol}

\newcommand{\Subitem}[1]{{\setlength\itemindent{12pt} \item[-] #1}}
\newcommand{\Subsubitem}[1]{{\setlength\itemindent{24pt} \item[○] #1}}
\newcommand{\Subsubsubitem}[1]{{\setlength\itemindent{36pt} \item[-] #1}}
\newcommand{\Subsubsubsubitem}[1]{{\setlength\itemindent{48pt} \item[○] #1}}

\title{Meeting 16: The blue-necked stormer}
\subtitle{Rudrá in Vedic \& post-Vedic literature}
\author{Nikhil Surya Dwibhashyam}
\date{12 June 2022}

\begin{document}

\frame{\titlepage}

\begin{frame} \frametitle{Meeting agenda}
\begin{itemize}
	\item (Re)introduction to discussion group
	\item Today's topic præsentation
	\item Free discussion
\end{itemize}
\end{frame}

\begin{frame} \frametitle{Why a Véda discussion group?}
\begin{itemize}
	\item What is Vedic literature?
	\Subitem Sáṁhitā-s
	\Subitem Brā́hmaṇa-s
	\Subitem Others?
	\item Véda-s vs.~Vedā́nta \& Upaniṣád-s
	\Subitem Later monism (ádvāita) vs.~earlier dualism (dvāitá)
	\item Modern (religious, scholarly) fixation upon later literature
\end{itemize}
\end{frame}

\begin{frame} \frametitle{What is there to discuss?}
\begin{itemize}
	\item History \& linguistics
	\item Metaphysics of religion
	\item Exegesis (e.g.~countering western narratives)
	\item Moral principles
	\item Nuances of Ṡrāutá ritual
\end{itemize}
\end{frame}

\begin{frame} \frametitle{Our plans}
\begin{itemize}
	\item Weekly meetings on Sundays at 1200 EDT
	\Subitem Followed by more in-depth literature meeting at 1330 EDT
	\item Different topic/aspect/angle discussed in depth every week
	\item Format:
	\Subitem Præsentation
	\Subitem Free-form discussion
	\item Eventually something more
\end{itemize}
\end{frame}

\begin{frame} \frametitle{WhatsApp group}
\begin{center}
	\href{https://chat.whatsapp.com/IXCQEkhfrcwHI7CNY8Fgat}{https://chat.whatsapp.com/IXCQEkhfrcwHI7CNY8Fgat}
\end{center}
\end{frame}

\begin{frame} \frametitle{Let's introduce ourselves!}
\begin{itemize}
	\item Name
	\item School (if student)
	\item How you found this group
	\item Briefly: background/interest in Hinduism \& Vedic literature
\end{itemize}
\end{frame}

\begin{frame}[label=questions] \frametitle{Some questions to consider}
\begin{itemize}
	\item What was Rudrá's role in the original Vedic religion?
	\item What were some epithets of His that were reinterpreted in post-Vedic times?
	\item What aspects of post-Purāṇá Ṡivá can be seen in Vedic Rudrá?
	\item When did the synthesis of Rudrá with Ṡivá occur historically, \& during what stage of literature is the shift evident?
	\item To what extent were Rudrá and Ṡivá still distinguished in later times?
	\item Who are the Rudrá-s really?
\end{itemize}
\end{frame}

\begin{frame} \frametitle{Who is Rudrá?}
\begin{itemize}
	\item External view
	\Subitem Etymology probably $\surd$\textit{rud}
	\Subsubitem Some say $\surd$\textit{vr̥dʰ}, $\surd$\textit{ru}, perhaps $\surd$\textit{rudʰ} (cf.~\textit{rudʰirá})
	\Subitem Name very often trisyllabic: \textit{Rudr̥á} or perhaps \textit{Rudará}
	\Subitem IE: very little connection
	\Subsubitem But father of the Marút-s: parellel to IE
	\Subsubitem Also: kóryos/Vrā́ta!
\end{itemize}
\end{frame}

\begin{frame} \frametitle{Who is Rudrá?}
\begin{itemize}
	\item Internal view
	\Subitem An important though not esp.~mentioned God in R̥V: 75 times
	\Subitem War-God, storm-God, wind-God, thunder-God, beast-God
	\Subitem Hymns in R̥V:
	\Subsubitem 1.114, 2.33, 7.46
	\Subitem Most frequently paired with Sóma
	\Subsubitem 1.43, 6.74
	\Subitem Plenty outside R̥V: famously TS 4.5
	\Subitem Later paired with Agní (AV+); acc.~Doniger, inherited features from Índra
\end{itemize}
\end{frame}

\begin{frame} \frametitle{Vedic descriptions of Rudrá}
\begin{itemize}
	\item Is brown/dark-red (R̥V 2.33.5, VS 16.7) \& shines like gold (R̥V 1.43.5)
	\Subitem Is blue-necked, \textit{nī́la-grīva}! (VS 16.7)
	\Subitem Has black belly \& red back (AV 2.27.6)
	\item Has braided hair (R̥V 1.114.5) \& is clothed in a skin (VS 3.61)
	\item Has firm limbs (R̥V 2.33.9)
	\item Is ``thousand-eyed'', prob.~metaphorical (AV 11.2.2)
	\item Takes many forms, \textit{puru-rū́pa} (R̥V 2.33.9)
	\item Inhabits the mountains (VS 16.2)
	\item Has 3 mothers, \textit{tri-ámbaka} (R̥V 7.59.12); later reinterpreted as 3 eyes
\end{itemize}
\end{frame}

\begin{frame} \frametitle{Vedic mythemes about Rudrá}
\begin{itemize}
	\item Carries thunderbolt \textit{didyút} as weapon, like Índra
	\Subitem But much more often: the bow \textit{dʰánvan} and arrows
	\item Beasts at the whim of weather: thus, \textit{paşu-páti} (AV+)
	\Subitem Also Himself bestial (R̥V 2.33.11), bull-like (2.33.7)
	\item Rules over the Heavens \& Earth (R̥V 7.46.2)
	\Subitem Father of existence (R̥V 6.49.10)
	\item Malevolent: man-slaying, \textit{nr̥hán} (R̥V 4.3.6)
	\Subitem Must pray to Him to avoid harm (R̥V 1.114.7); rather unique
	\Subsubitem Harm is from His lightning or sicknesses (AV 11.2)
	\item Remained behind when Gods ascended (ṠB 1.7.3.1)
	\item Receives remainder of sacrificial offerings (AD 2.4.2)
	\item Robber (VS 16.20)
	\item Best of all physicians, \textit{bʰiṣáktama} (R̥V 2.33.4)
	\Subitem (Usually physicians of the Gods are the Aşvín-s)
\end{itemize}
\end{frame}

\begin{frame} \frametitle{And who are the Rudrá-s?}
\begin{itemize}
	\item Class of Gods alongside Ādityá-s \& Vásu-s
	\item In my \& most western scholars' opinion: straightforwardly synonymous with Marút-s
	\Subitem who are also called Rudríya-s
	\Subitem But later indigenous sources disagree
	\Subsubitem cf.~general ``God as title'' phænomenon
	\Subsubitem or Puranic interpretation: Ṡivá splits into 11
	\item Mentioned from R̥V onwards
	\item Sons \& soldiers of \emph{the} Rúdra by the cow Pŕ̥şni
	\Subitem Interesting parellel to natural phænomena: storm-clouds
	\Subitem Literalist interpretation of Véda
	\item Associated with Índra in war rather than with Rudrá
\end{itemize}
\end{frame}

\begin{frame} \frametitle{Who is Ṡivá?}
\begin{itemize}
	\item Not a person!
	\Subitem Not controversial to anyone who has read the Véda
	\item Adjective: \glqq heilsam, heilbringend, hold, günstig\grqq
	\item Noun: \glqq Heil, Glück\grqq
	\item Used in R̥V to describe Agní, Índra, Tváṣṭr̥, Vāyú, i.a.
	\Subitem Rúdra only once! (10.92.9)
	\item Later association with Rúdra most likely euphemistic (cf.~Macdonell)
\end{itemize}
\end{frame}

\begin{frame} \frametitle{Rudrá in the Véda: an overview}
\begin{itemize}
	\item Terrible, fearsome, bestial, but divine \& not dæmonic
	\item Descriptions are consistent throughout Sáṁhitā-s (\& Brā́hmaṇa-s?)
	\item More emphasis is placed on Him towards later Sáṁhitā-s (esp.~VS)
	\Subitem VS description most conspicuously similar to later Ṡivá
	\item Morally neutral nature præsent from very beginning
	\Subitem But honored \& placated by men just as other Gods are
\end{itemize}
\end{frame}

\begin{frame} \frametitle{Rudrá after the Véda}
\begin{itemize}
	\item Almost always quā Ṡivá: little distinction
	\item TA \& KU have some of earliest consolidation
	\Subitem Umā \& Pārvatī
	\Subitem TA 10.22.1: definition of Rudrá as Ṡivá
	\item ``God of destruction'': not seen in Véda (nor is trinity, of course)
	\item Some similarities with Índra's downfall
	\Subitem Purāṇá-s portray Him as foolish
	\Subitem Constantly seeks Víṣṇu's help
	\Subitem Suffers personal/familial issues from His misdeeds
\end{itemize}
\end{frame}

\begin{frame} \frametitle{Rudrá outside Hinduism}
\begin{itemize}
	\item Ṡivá as a synthesis of Vedism with ``Dravidian'' beliefs
	\item ``Paşu-páti'' seal in IVC
	\item Lots of hogwash in my opinion
	\item Liṅga worship condemned in R̥V 7.21.5, \textit{şişná-deva}
	\item Buddhism: dæmon condemned to Hell, destroyed and converted to dʰamma
	\item Jainism: 11 Rudra-s
	\item Sikhism: mostly unrelated; name for divine order \textit{hukam}
\end{itemize}
\end{frame}

\begin{frame} \frametitle{The significance of Rudrá to Vedic revivalism}
\begin{itemize}
	\item Rudrá is the basis for the most important, supreme God of around a third of Hindus.
	\item Mythology \& aspects significantly different in Vedism vs.~Shaivism
	\item But much less of a dichotomy than Índra: Ṡivá continues to be worshiped \& honored under the name Rudrá with recognizably continuous traits
	\item Perhaps an opportunity for synthesis
\end{itemize}
\end{frame}

\againframe{questions}

\begin{frame} \frametitle{Some references}
\begin{itemize}
	\item Kazanas, N.~D. ``Indo-European deities \& the R̥g-vedá.'' \textit{J.~Indoeur.~Stud.}, \textbf{2001}.
	\item MacDonell, A.~A. ``Vedic Mythology.'' \textit{Grundr.~Indoar.~Philol.~Altertumskd.}, \textbf{1897}.
\end{itemize}
\end{frame}

\end{document}
