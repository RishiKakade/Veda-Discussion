\documentclass[pdf]{beamer}
\mode<presentation>{}
\usetheme{Rochester}
\usecolortheme{whale}
\beamertemplatenavigationsymbolsempty

\usepackage[ngerman]{babel}
\usepackage{fontspec}
\setmainfont[ItalicFont={brilli.ttf}, BoldFont={brillb.ttf}, Numbers=Lining]{brill.ttf}
\usefonttheme{serif}
\usepackage{microtype}
\usepackage{multicol}

\newcommand{\Subitem}[1]{{\setlength\itemindent{12pt} \item[-] #1}}
\newcommand{\Subsubitem}[1]{{\setlength\itemindent{24pt} \item[○] #1}}
\newcommand{\Subsubsubitem}[1]{{\setlength\itemindent{36pt} \item[-] #1}}
\newcommand{\Subsubsubsubitem}[1]{{\setlength\itemindent{48pt} \item[○] #1}}
\DeclareMathOperator*{\argmax}{arg\,max}

\title{Meeting 41: Boats, carts, chariots}
\subtitle{Transportation in the Vedic religion}
\author{Nikhil Surya Dwibhashyam}
\date{29 January 2023}

\begin{document}

\frame{\titlepage}

\begin{frame} \frametitle{Why a Véda discussion group?}
\begin{itemize}
	\item What is Vedic literature?
	\Subitem Sáṁhitā-s
	\Subitem Brā́hmaṇa-s
	\Subsubitem ĀŞS 24.1.31: Mántra--Brā́hmaṇayor Veda-nāma-dʰeyám
	\Subitem Others? What is Şrúti?
\end{itemize}
\end{frame}

\begin{frame} \frametitle{Why a Véda discussion group?}
\begin{itemize}
	\item Véda-s vs.~Upaniṣád-s, Itihāsá-s, Purāṇá-s, Vedāntá
	\Subitem AV 15.6.11--12?
	\Subitem Later philosophies like monism (ádvāita)
	\Subitem Moral--political subtexts
	\item Modern (religious, scholarly) fixation upon later literature
\end{itemize}
\end{frame}

\begin{frame} \frametitle{What is there to discuss?}
\begin{itemize}
	\item History \& linguistics
	\item Metaphysics of Vedic religion
	\item Exegetical narratives
	\Subitem Western vs.~Eastern
	\Subitem Scholarly ``vs.''~religious
	\item Moral principles
	\item Nuances of Şrāutá ritual
	\item Applying Vedic principles to our lives
\end{itemize}
\end{frame}

\begin{frame} \frametitle{Our plans}
\begin{itemize}
	\item Weekly ``general audience'' meetings on Sundays at 1000 EST
	\Subitem Followed by in-depth literature meeting at 1130 EST
	\item Different topic/aspect/angle discussed every week
	\item Format:
	\Subitem Præsentation
	\Subitem Free-form discussion
	\item Eventually something more
	\Subitem Poëtry composition
	\Subitem Children's books
	\Subitem Documentation
	\Subitem Actual rituals
\end{itemize}
\end{frame}

\begin{frame} \frametitle{WhatsApp group}
\begin{center}
	\href{https://chat.whatsapp.com/IXCQEkhfrcwHI7CNY8Fgat}{https://chat.whatsapp.com/IXCQEkhfrcwHI7CNY8Fgat}
\end{center}
\end{frame}

\begin{frame} \frametitle{Let's introduce ourselves!}
\begin{itemize}
	\item Briefly!
	\item Name
	\item School (if student) or where you're based
	\item How you found this group
	\item Background/interest in Hinduism \& Vedic literature
\end{itemize}
\end{frame}

\begin{frame} \frametitle{Contents}
\begin{itemize}
	\item Linguistics
	\item The vehicle
	\item Beasts of burden
	\Subitem The horse
	\Subitem The ox
	\Subitem The ass
	\item The cart
	\item The chariot
	\item Chariot-racing
	\item The boat
	\item Smoke
	\item Teleportation
\end{itemize}
\end{frame}

\begin{frame} \frametitle{Linguistics}
\begin{block} {$\surd$\textit{vah} (MW on R̥V)}
\begin{itemize}
	\item to carry, transport, convey (with instr.~of vehicle)
	\item to lead, conduct (esp.~offerings to the gods, said of Agni)
	\item to bear along (water, said of rivers)
	\item to draw (a car), guide (horses \&c.)
\end{itemize}
\end{block}
\end{frame}

\begin{frame} \frametitle{Linguistics}
\begin{itemize}
	\item \textit{Vā́hana}: AB; ifc.~in R̥V
	\item \textit{Ánas}
	\Subitem {\textit{Anaḍ-váh}}
	\item Side note: \textit{su-kʰá} (VS \& later)
\end{itemize}
\end{frame}

\begin{frame} \frametitle{The vehicle}
\begin{itemize}
	\item Lack of later, unique vā́hana-s for each God
	\item Instead: invariably horses, oxen, chariots
	\Subitem {\textit{Váhni}}
\end{itemize}
\end{frame}

\begin{frame} \frametitle{The horse}
\begin{itemize}
	\item \textit{Áşva}
	\item The second-most prized and sacred animal
	\item R̥V 1.163.10 (Griffith):
	\Subitem Symmetrical in flank, with rounded haunches, mettled like heroes, the Celestial Coursers put forth their strength, like swans in lengthened order, when they, the Steeds, have reached the heavenly causeway.
	\item Innumerable descriptors: \textit{vājín}, \textit{āşú}, \textit{árvant}, \&c.
	\Subitem Battle!
	\item Prohibition on consumption?
\end{itemize}
\end{frame}

\begin{frame} \frametitle{The horse in mythology}
\begin{itemize}
	\item Most famously: Aşvín-s
	\item Hárī
	\item Sū́rya's horses
	\item Many important horses, some named (e.g.~Pāidvá)
	\Subitem Dadʰikrā́?
	\item Lots of other imagery
\end{itemize}
\end{frame}

\begin{frame} \frametitle{The ox}
\begin{itemize}
	\item {\textit{Anaḍ-váh}}, \textit{gó}, \&c.
	\item Castration: \textit{vádʰri} vs.~\textit{vŕ̥ṣan}
	\item Cart of Uṣás
	\item Generic use as draft-animal
	\item Prohibition on consumption?
\end{itemize}
\end{frame}

\begin{frame} \frametitle{The ass}
\begin{itemize}
	\item \textit{Gardabʰá}, \textit{aşvatará} (mule)
	\item Inferior connotations
	\item cf.~R̥V 1.29.5, 3.53.23
	\item Inauspicious or noisome cry
	\item But still valuable: given as gifts (R̥V 8.56.3)
	\item \textit{Dví-retas} (TS 5.1.5.5)
\end{itemize}
\end{frame}

\begin{frame} \frametitle{The cart}
\begin{itemize}
	\item Draft-wagon: distinguished from chariot!
	\Subitem But \textit{anar-víş} (R̥V 1.121.7)
	\item Uṣás
	\item Bridal wagon in marriage-hymn
	\Subitem Covering (\textit{chadís})
\end{itemize}
\end{frame}

\begin{frame} \frametitle{Chariot-racing}
\begin{itemize}
	\item One of only two games mentioned in the Véda
	\item Præstigious sport
	\Subitem Exorbitant prizes won
	\item R̥V 10.102: Múdgala \& Mudgalā́nī
	\Subitem Metaphors in hymn
\end{itemize}
\end{frame}

\begin{frame} \frametitle{The boat}
\begin{itemize}
	\item River-crossing
	\item Not very advanced (``ships''?)
	\item Lots of imagery
	\Subitem R̥V 2.42.1 on the bird's voice
\end{itemize}
\end{frame}

\begin{frame} \frametitle{Smoke \& teleportation}
\begin{itemize}
	\item Agní as messenger (\textit{dūtá})
	\item Transport to Svargá
	\item Ápsaras-es
	\item Devá-s
	\Subitem Transmutation (e.g.~horses)
\end{itemize}
\end{frame}

\begin{frame} \frametitle{References}
\begin{itemize}
	\item Gode, P.~K., \textbf{1946}. ``Carriage-manufacture in the Vedic period \& in ancient China in 1121 B.C.'' Ann.~Bhandarkar Orient.~Res.~Inst.
	\item MacDonell, A.~A., Keith, A.~B., \textbf{1912}. \textit{The Vedic index of names \& subjects.}
\end{itemize}
\end{frame}

\end{document}
