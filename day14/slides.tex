\documentclass[pdf]{beamer}
\mode<presentation>{}
\usetheme{Rochester}
\usecolortheme{whale}
\beamertemplatenavigationsymbolsempty

\usepackage{fontspec}
\setmainfont[ItalicFont={brilli.ttf}, BoldFont={brillb.ttf}, BoldItalicFont={brillbi.ttf}]{brill.ttf}
\usefonttheme{serif}
\usepackage{microtype}
\usepackage{multicol}

\newcommand{\Subitem}[1]{{\setlength\itemindent{12pt} \item[-] #1}}
\newcommand{\Subsubitem}[1]{{\setlength\itemindent{24pt} \item[○] #1}}
\newcommand{\Subsubsubitem}[1]{{\setlength\itemindent{36pt} \item[-] #1}}
\newcommand{\Subsubsubsubitem}[1]{{\setlength\itemindent{48pt} \item[○] #1}}

\title{Meeting 14: Vr̥trahán Ánedya}
\subtitle{Índra in Vedic \& post-Vedic literature}
\author{Nikhil Surya Dwibhashyam}
\date{29 May 2022}

\begin{document}

\frame{\titlepage}

\begin{frame} \frametitle{Meeting agenda}
\begin{itemize}
	\item (Re)introduction to discussion group
	\item Today's topic præsentation
	\item Free discussion
\end{itemize}
\end{frame}

\begin{frame} \frametitle{Why a Véda discussion group?}
\begin{itemize}
	\item What is Vedic literature?
	\Subitem Sáṁhitā-s
	\Subitem Brā́hmaṇa-s
	\Subitem Others?
	\item Véda-s vs.~Vedā́nta \& Upaniṣád-s
	\Subitem Later monism (ádvāita) vs.~earlier dualism (dvāitá)
	\item Modern (religious, scholarly) fixation upon later literature
\end{itemize}
\end{frame}

\begin{frame} \frametitle{What is there to discuss?}
\begin{itemize}
	\item History \& linguistics
	\item Metaphysics of religion
	\item Exegesis (e.g.~countering western narratives)
	\item Moral principles
	\item Nuances of Ṡrāutá ritual
\end{itemize}
\end{frame}

\begin{frame} \frametitle{Our plans}
\begin{itemize}
	\item Weekly meetings on Sundays at 1200 EDT
	\Subitem Followed by more in-depth literature meeting at 1330 EDT
	\item Different topic/aspect/angle discussed in depth every week
	\item Format:
	\Subitem Præsentation
	\Subitem Free-form discussion
	\item Eventually something more
\end{itemize}
\end{frame}

\begin{frame} \frametitle{WhatsApp group}
\begin{center}
	\href{https://chat.whatsapp.com/IXCQEkhfrcwHI7CNY8Fgat}{https://chat.whatsapp.com/IXCQEkhfrcwHI7CNY8Fgat}
\end{center}
\end{frame}

\begin{frame} \frametitle{Let's introduce ourselves!}
\begin{itemize}
	\item Name
	\item School (if student)
	\item How you found this group
	\item Briefly: background/interest in Hinduism \& Vedic literature
\end{itemize}
\end{frame}

\begin{frame}[label=questions] \frametitle{Some questions to consider}
\begin{itemize}
	\item What was Índra's role in the original Vedic religion?
	\Subitem What aspects of His were subsumed by other Gods in later Hinduism?
	\item What are the chief contradictions between the Vedic \& the post-Vedic depictions of Índra?
	\item When did this shift in views occur historically, \& during what stage of literature is the shift evident?
	\item What philosophical motivations could lie behind the denigration of Índra?
	\Subitem political motivations?
	\item How might exposing this uniquely post-Vedic denigration of Índra be beneficial to a goal of reviving Vedic traditions?
\end{itemize}
\end{frame}

\begin{frame} \frametitle{Who is Índra?}
\begin{itemize}
	\item External view
	\Subitem Etymology unknown (or nonexistent): perhaps $\surd$\textit{ind}
	\Subitem Name very often trisyllabic: \textit{Índr̥a} or perhaps \textit{Índara}
	\Subitem Equivalent of sky-father in other IE religions, but not himself the Sky-Father
	\Subitem Inimical to the Avestan faith (Vd.~10.9: \textit{paiti-pərəne Iṇdrəm})
	\Subitem Mentioned in Hittite--Mitanni treaty (14$^{\text{th}}$ c.~BCE): \textit{Indara}, \textit{Intar}
\end{itemize}
\end{frame}

\begin{frame} \frametitle{Who is Índra?}
\begin{itemize}
	\item Internal view
	\Subitem Most important (e.g.~mentioned) God in R̥g-vedá
	\Subitem Born of Áditi; married to Indrāṇī́
	\Subitem {Epithets: \textit{Vr̥trahán}, \textbf{\textit{ánedya}}, \textit{magʰávan}, \textit{vajrín}/\textit{vájrivan}, \textit{şakrá} (cf.~Buddhism), \textit{şatá-kratu}}
	\Subitem Has golden skin, hair, beard, horses, weapon (R̥V 10.96)
	\Subitem The One King (\textit{eka-rā́j}): R̥V 8.37.3
	\Subsubitem But follows the law of Savitŕ̥ (R̥V 2.38.9)
	\Subitem Most frequently paired with Agní; Víṣṇu is his helper
\end{itemize}
\end{frame}

\begin{frame} \frametitle{Vedic mythemes about Índra}
\begin{itemize}
	\item Drank thirty seas of sóma as a newborn (R̥V 8.77.4)
	\item Slew Vr̥trá \& released the waters (R̥V 2.11.2)
	\item Freed kine from the cave Valá (R̥V 2.24.3)
	\item Crushed the Dawn's chariot (R̥V 2.15.6)
	\item Slew the Dásyu-s \& protected the Ā́rya-s (R̥V 3.34.9)
	\item Mocked Vr̥ṣā́-kapi (R̥V 10.86)
	\item Gave golden skin to Apālā́ (R̥V 8.91)
	\item Slew Árbuda with snow (R̥V 8.32.6)
	\item \& countless more
\end{itemize}
\end{frame}

\begin{frame} \frametitle{Índra in the Véda}
\begin{itemize}
	\item Blemishless
	\item Uncriticized
	\item Primordial
	\item Supreme (though not omnipotent)
	\Subitem As opposed to Prajā́pati, Víṣṇu, or Rudrá
	\item God of all 3 estates
\end{itemize}
\end{frame}

\begin{frame} \frametitle{Índra after the Véda}
\begin{itemize}
	\item Hedonistic
	\Subitem Drunken
	\Subitem Promiscuous \& adulterous
	\item Ally but nuisance to more supreme God (e.g. Víṣṇu)
	\item Different Índra every æra (Manv-antara system)
	\item Frequently cast out of heavens, humiliated
	\Subitem Even by sages \& animals
	\item Political association with Kṣatríya-s
	\item Nonetheless:
	\Subitem Still revered for strength \& rulership, e.g.~metaphorical \textit{nar'-endra}
	\Subitem Still worshiped in many rituals
\end{itemize}
\end{frame}

\begin{frame} \frametitle{An example of post-Vedic misinterpretation}
\begin{itemize}
	\item \textit{Sahásra-muṣka}: ``having thousand testicles''
	\Subitem Metaphor for virility
	\Subitem Used for Índra (R̥V 6.46.3), also Agní (R̥V 8.19.32)
	\Subitem Also cf.~\textit{sahasr'-ākṣá} (R̥V 1.23.3)
	\item \textit{Muṣká}: unambiguously ``testicle'' in origin (little mouse)
	\Subitem But AV, TS, \& beyond: also ``vulva''
	\item Thus: ``covered in thousand vulvæ''
	\Subitem Integrated with Ahalyā story, or with others
	\Subsubitem Itself misinterpretation of ṠB 3.3.4.18: \textit{áhalyāyāi jārá}
	\Subitem Sometimes: becomes \textit{klībá}
	\item In sum: transition from supreme masculinity to total humiliation
\end{itemize}
\end{frame}

\begin{frame} \frametitle{The chronology of the Índra shift}
\begin{itemize}
	\item Some (e.g.~Söhnen) claim shift began with Brā́hmaṇa-s
	\Subitem Not necessarily true, at least until very late: cf.~Jamison on \textit{amedʰyá}
	\item Shift certainly began by time of (Vāiṣṇavá) epics
	\Subitem Ahalyā story found in both R \& MBʰ
	\item Shift complete by time of Purāṇá-s
	\Subitem Upaniṣád-s?
	\Subitem Outside Hinduism?
\end{itemize}
\end{frame}

\begin{frame} \frametitle{The significance of Índra to Vedic revivalism}
\begin{itemize}
	\item To reiterate: \emph{the} most important God
	\item Those who accept the Véda-s as supreme must necessarily reject post-Vedic interpretations of His role.
	\item Índra: symbol of Vedic orthopraxy vs.~mainstream, post-Vedic Hinduism
	\item An \textit{Āindrá} movement?
\end{itemize}
\end{frame}

\againframe{questions}

\begin{frame} \frametitle{Some interesting papers \& articles}
\begin{itemize}
	\item Fournet, A. ``About the Mitanni--Aryan gods.'' \textit{J.~Indoeur.~Stud.}, \textbf{2010}.
	\item Jamison, S.~W. ``The ravenous hyenas \& the wounded Sun.'' \textbf{1991}.
	\item Kazanas, N.~D. ``Indo-European deities \& the R̥g-vedá.'' \textit{J.~Indoeur.~Stud.}, \textbf{2001}.
	\item Söhnen, R. ``Indra \& women.'' \textit{Bull.~Sch.~Orient.~Afr.~Stud.}, \textbf{1991}.
	\item Venkatasubbiah, A. ``On Indra's winning of cows \& waters.'' \textit{Z.~Dtsch.~Morgenl.~Ges.}, \textbf{1965}.
\end{itemize}
\end{frame}

\end{document}
