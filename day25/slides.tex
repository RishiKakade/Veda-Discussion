\documentclass[pdf]{beamer}
\mode<presentation>{}
\usetheme{Rochester}
\usecolortheme{whale}
\beamertemplatenavigationsymbolsempty

\usepackage[ngerman]{babel}
\usepackage{fontspec}
\setmainfont[ItalicFont={brilli.ttf}, BoldFont={brillb.ttf}, Numbers=Lining]{brill.ttf}
\usefonttheme{serif}
\usepackage{microtype}
\usepackage{multicol}

\newcommand{\Subitem}[1]{{\setlength\itemindent{12pt} \item[-] #1}}
\newcommand{\Subsubitem}[1]{{\setlength\itemindent{24pt} \item[○] #1}}
\newcommand{\Subsubsubitem}[1]{{\setlength\itemindent{36pt} \item[-] #1}}
\newcommand{\Subsubsubsubitem}[1]{{\setlength\itemindent{48pt} \item[○] #1}}

\title{Meeting 24: Hymns, songs, \& lineage}
\subtitle{The Vedic priesthood}
\author{Nikhil Surya Dwibhashyam}
\date{2 October 2022}

\begin{document}

\frame{\titlepage}

\begin{frame} \frametitle{Why a Véda discussion group?}
\begin{itemize}
	\item What is Vedic literature?
	\Subitem Sáṁhitā-s
	\Subitem Brā́hmaṇa-s
	\Subitem Others?
	\item Véda-s vs.~Vedā́nta, Upaniṣád-s, Purāṇá-s
	\Subitem Later monism (ádvāita) vs.~earlier dualism
	\Subitem Moral--political subtexts
	\item Modern (religious, scholarly) fixation upon later literature
\end{itemize}
\end{frame}

\begin{frame} \frametitle{What is there to discuss?}
\begin{itemize}
	\item History \& linguistics
	\item Metaphysics of religion
	\item Exegesis (e.g.~western vs.~Eastern narratives)
	\item Moral principles
	\item Nuances of Ṡrāutá ritual
	\item How we can apply Vedic principles to our lives
\end{itemize}
\end{frame}

\begin{frame} \frametitle{Our plans}
\begin{itemize}
	\item Weekly ``general audience'' meetings on Sundays at 1200 EDT
	\Subitem Followed by in-depth literature meeting at 1330 EDT
	\item Different topic/aspect/angle discussed every week
	\item Format:
	\Subitem Præsentation
	\Subitem Free-form discussion
	\item Eventually something more
	\Subitem Children's books
	\Subitem Poëtry collective
	\Subitem Documentation
	\Subitem Actual rituals
\end{itemize}
\end{frame}

\begin{frame} \frametitle{WhatsApp group}
\begin{center}
	\href{https://chat.whatsapp.com/IXCQEkhfrcwHI7CNY8Fgat}{https://chat.whatsapp.com/IXCQEkhfrcwHI7CNY8Fgat}
\end{center}
\end{frame}

\begin{frame} \frametitle{Let's introduce ourselves!}
\begin{itemize}
	\item Briefly!
	\item Name
	\item School (if student)
	\item How you found this group
	\item Background/interest in Hinduism \& Vedic literature
\end{itemize}
\end{frame}

\begin{frame}[label=questions] \frametitle{Some questions to consider}
\begin{itemize}
	\item What is the role of the clergy in Vedic society? in modern societies?
	\item How might the Vedic priesthood have evolved and developed over time?
	\Subitem in response to ritualistic development?
	\Subitem in response to political pressures?
	\item To what extent is the Vedic priesthood alive today?
	\Subitem What benefits are there to ``reviving'' it?
	\Subitem How can it be revived?
\end{itemize}
\end{frame}

\begin{frame} \frametitle{What is a priest?}
\begin{itemize}
	\item A conduit between Gods \& other men
	\Subitem Or a ``messenger'', if you will
	\Subitem AB 8.24: ``Verily, the Gods do not eat the food offered by the king who is without a puró-hita.''
	\Subitem Agní Himself is a priest (r̥tvij), the conduit (dūtá)
	\Subsubitem between Gods \& men
	\Subsubitem between Gods
	\item Protector of the natural order (go-pā́ R̥tásya)
	\Subitem Jamison: ``conjurer'' of the Truth by his ``formulation'' (bráhman)
	\item Beseecher for material, worldly success
\end{itemize}
\end{frame}

\begin{frame} \frametitle{What is a priest's role?}
\begin{itemize}
	\item In IE society: the first estate (V1)
	\Subitem {Placed at the front: puró-hita}
	\Subitem {\emph{Joint rule} with V2: bráhma--kṣatrá}
	\Subsubitem Over the freemen \& serfs
	\item Many attributes not unique to Vedic or IE society
	\Subitem Patrilineal descent: Lēvīyyīm, Kōhănīm
	\Subitem Association with (e.g.~wearing) white: Christianity
	\Subitem Much later: emphasis on celibacy
	\item Priestly conceit: brahma-varcasá
\end{itemize}
\end{frame}

\begin{frame} \frametitle{The priest as bard}
\begin{itemize}
	\item Traveling priests in a nomadic society: esp.~war-bands
	\Subitem Besought the Gods for success in war
	\Subitem Praised the Ā́rya war-heroes and kings
	\Subitem {In return: gifted prizes of kine, gold, slaves (cf.~dāna-stutí-s)}
	\item \emph{2-way interaction} with the Vedic texts
	\Subitem Ongoing composition of ``newest hymn'' by seers (ŕ̥ṣi-s)
	\Subitem {\emph{Inventive} poëtic devices, variances in meter \& language}
	\Subitem Constant collation, organization, commentary, analysis
	\item Bardic composition reduced in post-Vedic æra
\end{itemize}
\end{frame}

\begin{frame} \frametitle{The priest as custodian}
\begin{itemize}
	\item Custodian of spiritual lineage
	\Subitem Passed down the Véda over millennia
	\Subitem Maintainers of ritual orthopraxy
	\Subitem Teaching: important part of priest's role
	\item Custodian of \emph{literal} lineage
	\Subitem All priests descended patrilineally from Vedic seers (cf.~gotrá)
	\Subitem System of agnatic descent as model for general Ā́rya society
	\Subsubitem Traditions of ancestor-worship 
	\item Custodian of morality (profane orthopraxy)
\end{itemize}
\end{frame}

\begin{frame} \frametitle{The 4-fold Vedic division}
\begin{itemize}
	\item All priests are Brāhmaṇá-s: -a derivation from \textit{brahmán}
	\item Brahmán
	\Subitem Sometimes generic term for priest, synonym for Brāhmaṇá or r̥tvij
	\Subitem Overseer of the sacrifice
	\Subitem Corrects any mistakes made
	\item Hótr̥
	\Subitem In one sense the ``main'' priest
	\Subitem Recites hymns of praise and prayer to the Gods
\end{itemize}
\end{frame}

\begin{frame} \frametitle{The 4-fold Vedic division (contd.)}
\begin{itemize}
	\item Ud-gātŕ̥
	\Subitem {\emph{Sings} hymns of praise in the background}
	\Subitem Cascading melodies with ``lullaby''-like sacred particles
	\Subitem Originally: had instrumental assistants in the background?
	\item Adʰvaryú 
	\Subitem In another sense the ``main'' priest
	\Subitem Performs most of the physical actions of the sacrifice
	\item Keith: ``a superabundance of technical terms''
	\Subitem Pótr̥, praşāstŕ̥, \&c.
\end{itemize}
\end{frame}

\begin{frame} \frametitle{4 priest \& 4 Véda-s}
\begin{itemize}
	\item Straightforward associations:
	\Subitem The hótr̥ recites hymns from the Véda of hymns (R̥g-vedá)
	\Subitem The ud-gātŕ̥ sings songs from the Véda of songs (Sāma-vedá)
	\Subitem The adʰvaryú chants formulæ from the Véda of formulæ (Yajur-vedá)
	\item The brahmán later becomes associated with the Atʰarva-vedá
	\Subitem Not an absurd connection: ``healing'' of sacrifice with magic
\end{itemize}
\end{frame}

\begin{frame} \frametitle{Auxiliaries in the sacrifice}
\begin{itemize}
	\item Detailed system of assistants outlined in Ṡrāuta-sū́tra-s (AṠS, ṠṠS):
	\Subitem Assistants of hótr̥: maitrāvaruná, achāvāká, grāvastút
	\Subitem Assistants of ud-gātŕ̥: prastotŕ̥, pratihartŕ̥, subrahmaṇyá
	\Subitem Assistants of adʰvaryú: pratiprastʰātŕ̥, néṣṭr̥, unnetŕ̥
	\Subitem Assistants of brahmán: brāhmanāchamsín, agnī́dʰ, pótr̥
	\item Many likely not præsent in early (or even later) Vedic times!
	\item Rituals generally smaller-scale and less elaborate
\end{itemize}
\end{frame}

\begin{frame} \frametitle{Developments}
\begin{itemize}
	\item Clear evolution of ritual \& priesthood
	\item Ritual becomes more formalized \& fossilized
	\item Early Vedic: Gods can appear in much more ordinary circumstances
	\Subitem Rituals shorter, more localized with fewer people
	\item Lexical reïnterpretation (viz.~puró-hita, krátu)
	\item The ``Brahmanical ascendancy'' model
	\Subitem Greater dependence $\rightarrow$ greater privileges
	\item Often associated with the overall (d)evolution of Hindu morality
\end{itemize}
\end{frame}

\againframe{questions}

\end{document}
