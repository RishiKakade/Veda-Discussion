\documentclass[pdf]{beamer}
\mode<presentation>{}
\usetheme{Rochester}
\usecolortheme{whale}
\beamertemplatenavigationsymbolsempty

\usepackage[ngerman]{babel}
\usepackage{fontspec}
\setmainfont[ItalicFont={brilli.ttf}, BoldFont={brillb.ttf}, Numbers=Lining]{brill.ttf}
\usefonttheme{serif}
\usepackage{microtype}
\usepackage{multicol}

\newcommand{\Subitem}[1]{{\setlength\itemindent{12pt} \item[-] #1}}
\newcommand{\Subsubitem}[1]{{\setlength\itemindent{24pt} \item[○] #1}}
\newcommand{\Subsubsubitem}[1]{{\setlength\itemindent{36pt} \item[-] #1}}
\newcommand{\Subsubsubsubitem}[1]{{\setlength\itemindent{48pt} \item[○] #1}}
\DeclareMathOperator*{\argmax}{arg\,max}

\title{Meeting 28: The Devá-s \& Ásura-s}
\subtitle{Contradiction or subsumption?}
\author{Nikhil Surya Dwibhashyam}
\date{23 October 2022}

\begin{document}

\frame{\titlepage}

\begin{frame} \frametitle{Why a Véda discussion group?}
\begin{itemize}
	\item What is Vedic literature?
	\Subitem Sáṁhitā-s
	\Subitem Brā́hmaṇa-s
	\Subitem Others?
	\item Véda-s vs.~Vedā́nta, Upaniṣád-s, Purāṇá-s
	\Subitem Later monism (ádvāita) vs.~earlier dualism
	\Subitem Moral--political subtexts
	\item Modern (religious, scholarly) fixation upon later literature
\end{itemize}
\end{frame}

\begin{frame} \frametitle{What is there to discuss?}
\begin{itemize}
	\item History \& linguistics
	\item Metaphysics of religion
	\item Exegesis (e.g.~western vs.~Eastern narratives)
	\item Moral principles
	\item Nuances of Ṡrāutá ritual
	\item How we can apply Vedic principles to our lives
\end{itemize}
\end{frame}

\begin{frame} \frametitle{Our plans}
\begin{itemize}
	\item Weekly ``general audience'' meetings on Sundays at 1200 EDT
	\Subitem Followed by in-depth literature meeting at 1330 EDT
	\item Different topic/aspect/angle discussed every week
	\item Format:
	\Subitem Præsentation
	\Subitem Free-form discussion
	\item Eventually something more
	\Subitem Children's books
	\Subitem Poëtry collective
	\Subitem Documentation
	\Subitem Actual rituals
\end{itemize}
\end{frame}

\begin{frame} \frametitle{WhatsApp group}
\begin{center}
	\href{https://chat.whatsapp.com/IXCQEkhfrcwHI7CNY8Fgat}{https://chat.whatsapp.com/IXCQEkhfrcwHI7CNY8Fgat}
\end{center}
\end{frame}

\begin{frame} \frametitle{Let's introduce ourselves!}
\begin{itemize}
	\item Briefly!
	\item Name
	\item School (if student)
	\item How you found this group
	\item Background/interest in Hinduism \& Vedic literature
\end{itemize}
\end{frame}

\begin{frame}[label=questions] \frametitle{Some questions to consider}
\begin{itemize}
	\item Who are the Devá-s?
	\item Who are the Ásura-s?
	\item What parallels do They have in other IE religions?
	\item Are \textit{Devá} and \textit{Ásura} synonyms? antonyms? neither?
	\item To what extent is the conflict between Devá-s \& Ásura-s a later development?
\end{itemize}
\end{frame}

\begin{frame} \frametitle{Etymology}
\begin{itemize}
	\item \textit{Devá} < $\surd$\textit{div} ``shine'' or \textit{div} ``Heavens''
	\Subitem {\textit{divyá}}
	\item \textit{Ásura} < \textit{ásu} ``life force''
	\Subitem R̥V 10.15.1: Pitŕ̥-s attained ásu
	\Subitem Uṇādi: < $\surd$\textit{as} ``cast''
	\Subitem Many false etymologies
	\Subsubitem I recently heard < \textit{súrā}!
	\Subitem Obviously cognate with Av.~\textit{Ahura}
	\Subsubitem Æsir
	\Subitem {\textit{asurī́a}, \textit{asuryá}}
\end{itemize}
\end{frame}

\begin{frame} \frametitle{Some western theories}
\begin{itemize}
	\item Parpola, Shendge: Devá-s for Ā́rya-s, Ásura-s for Dásyus
	\item Delhi: PII God named Ásura > Váruṇa
	\item Ásuras repræsenting older Gods like Dív Pitŕ̥
	\Subitem Agní ``leaving'' Father for Devá-s: R̥V 10.124.2--3
	\item Consensus: Initially (quasi-)synonymous, then conflict developed
	\item I argue: all of these are wrong.
\end{itemize}
\end{frame}

\begin{frame} \frametitle{Apposition}
\begin{itemize}
	\item \textit{Yákṣvā mahé • sāumanasā́ya Rudrám. / Námobʰir Devám • Ásuraṁ duvasya} (R̥V 5.42.11).
	\Subitem ``Worship Rudrá for His great goodwill.  With salutations honor the God, Ásura.''
	\item \textit{Mahā́ntā Mitrā́--Váruṇā / samrā́jā Devā́v--Ásurā} (R̥V 8.25.4).
	\Subitem ``Great Mitrá--Váruṇa, sovereign God--Ásura-s.''
	\item \textit{Tā́ hí Devā́nām • Ásurā tā́v aryā́} (R̥V 7.65.2).
	\Subitem ``For they [Mitrá--Váruṇa] are Ásura-s of the Gods, those kind ones.''
	\item \textit{Mahád Devā́nām • Asuratvám ékam} (R̥V 3.55.1).
	\Subitem ``Great is the sole Ásura-hood of the Gods.''
\end{itemize}
\end{frame}

\begin{frame} \frametitle{All Gods given label Ásura}
\begin{itemize}
	\item Præviously: Rudrá, Mitrá, Váruṇa
	\item \textit{Tuáṁ rā́jā • Indr̥a yé ca Devā́ḥ. / Rákṣā nr̥̄́n.  Pāhí • Asura Tvám asmā́n} (R̥V 1.174.1).
	\Subitem ``Thou art king, O Índra, of all the Gods.  Protect the men.  Guard us, O thou Ásura.''
	\item Others: Savitŕ̥, Dív Pitŕ̥, Agní, Pūṣán, Aryamán, Ādityá-s
\end{itemize}
\end{frame}

\begin{frame} \frametitle{Evil Ásura-s in the R̥g-vedá!}
\begin{itemize}
	\item Not a ``later interpolation''
	\Subitem Spread throughout both old and new hymns!
	\item R̥V 2.30.4, 7.99.5: heroes of the Ásura
	\item R̥V 8.96.9, 10.53.4, 10.157.4: Ásura-s Themselves
\end{itemize}
\end{frame}

\begin{frame} \frametitle{Subsumption theory}
\begin{itemize}
	\item Not difficult to synthesize views
	\item $D \in A$
	\Subitem {$A = \argmax_{D} \mathsf{Asuratvá}$}
	\Subitem Separately: $A = x : x \in A \land x \notin D$
\end{itemize}
\end{frame}

\begin{frame} \frametitle{Actual later developments}
\begin{itemize}
	\item Some synonymous use with Rákṣas in Classical period
	\item (Incomplete) loss of original meaning synonymous with Devá-s
\end{itemize}
\end{frame}

\againframe{questions}

\end{document}
