\documentclass[pdf]{beamer}
\mode<presentation>{}
\usetheme{Rochester}
\usecolortheme{whale}
\beamertemplatenavigationsymbolsempty

\usepackage{fontspec}
\setmainfont[ItalicFont={brilli.ttf}, BoldFont={brillb.ttf}]{brill.ttf}
\usefonttheme{serif}
\usepackage{microtype}
\usepackage{multicol}

\newcommand{\Subitem}[1]{{\setlength\itemindent{12pt} \item[-] #1}}
\newcommand{\Subsubitem}[1]{{\setlength\itemindent{24pt} \item[○] #1}}
\newcommand{\Subsubsubitem}[1]{{\setlength\itemindent{36pt} \item[-] #1}}
\newcommand{\Subsubsubsubitem}[1]{{\setlength\itemindent{48pt} \item[○] #1}}

\title{Meeting 13: From úc-cʰiṣṭa to pra-sādá}
\subtitle{Food \& diet in the Vedic religion}
\author{Nikhil Surya Dwibhashyam}
\date{22 May 2022}

\begin{document}

\frame{\titlepage}

\begin{frame} \frametitle{Meeting agenda}
\begin{itemize}
	\item (Re)introduction to discussion group
	\item Today's topic præsentation
	\item Free discussion
\end{itemize}
\end{frame}

\begin{frame} \frametitle{Why a Véda discussion group?}
\begin{itemize}
	\item What is Vedic literature?
	\Subitem Sáṁhitā-s
	\Subitem Brā́hmaṇa-s
	\Subitem Others?
	\item Véda-s vs.~Vedā́nta \& Upaniṣád-s
	\Subitem Later monism (ádvāita) vs.~earlier dualism (dvāitá)
	\item Modern (religious, scholarly) fixation upon later literature
\end{itemize}
\end{frame}

\begin{frame} \frametitle{What is there to discuss?}
\begin{itemize}
	\item History \& linguistics
	\item Metaphysics of religion
	\item Exegesis (e.g.~countering western narratives)
	\item Moral principles
	\item Nuances of Ṡrāutá ritual
\end{itemize}
\end{frame}

\begin{frame} \frametitle{Our plans}
\begin{itemize}
	\item Weekly meetings on Sundays at 1200 EDT
	\Subitem Followed by more in-depth literature meeting at 1330 EDT
	\item Different topic/aspect/angle discussed in depth every week
	\item Format:
	\Subitem Præsentation
	\Subitem Free-form discussion
	\item Eventually something more
\end{itemize}
\end{frame}

\begin{frame} \frametitle{WhatsApp group}
\begin{center}
	\href{https://chat.whatsapp.com/IXCQEkhfrcwHI7CNY8Fgat}{https://chat.whatsapp.com/IXCQEkhfrcwHI7CNY8Fgat}
\end{center}
\end{frame}

\begin{frame} \frametitle{Let's introduce ourselves!}
\begin{itemize}
	\item Name
	\item School (if student)
	\item How you found this group
	\item Briefly: background/interest in Hinduism \& Vedic literature
\end{itemize}
\end{frame}

\begin{frame}[label=questions] \frametitle{Some questions to consider}
\begin{itemize}
	\item What was the spiritual significance of food? of sacrificing it?
	\item How did materialistic concerns about food interact with spiritual concerns? Is this dichotomy valid for the Vedic religion?
	\item What sort of diet did the Vedic Ā́rya-s have? What foods were most beloved by them?
	\item What food is morally prohibited in the Vedic religion?
\end{itemize}
\end{frame}

\begin{frame} \frametitle{What is food's significance?}
\begin{itemize}
	\item Central to Vedic metaphysics
	\Subitem {\textit{Annādáş ca vā́ idaṁ sárvam ánnaṁ ca}: ``The food-eater \& the food are everything here.'' (ṠB 11.1.16.19)}
	\Subitem {\textit{Anna-jīvanáṁ h' īdaṁ sárvam}: ``Everything here lives on food.'' (ṠB 7.5.1.20)}
	\Subitem Eating as conquest: Índra \& Vr̥trá (ṠB 1.6.3.17)
	\Subsubitem Food \& society: attŕ̥ vs.~ádya
	\item Sacrifice \& reciprocality
	\Subitem Link between the mundane \& the heavenly, carried through Agní (dūtá)
\end{itemize}
\end{frame}

\begin{frame} \frametitle{What is food's significance?}
\begin{itemize}
	\item Of course, central to human relations
	\Subitem Material realities: famine \&c.
	\Subitem {\textit{Paro-gavyūtí • ánirām ápa kṣúdʰam}}: ``Drive away weakness \& hunger to the distant plains!'' (R̥V 8.60.20)
	\Subitem Gift-giving, generosity, enmity
\end{itemize}

\begin{center}
	\textit{Ná sá sákʰā • yó ná dádāti sákʰye}

	\textit{sacā-bʰúve • sácamānāya pitváḥ.}
	
	He is not a friend who giveth not to his friend,

	to his comrade who pursues food.

	---R̥V 10.117.4
\end{center}
\end{frame}

\begin{frame} \frametitle{The role of food in sacrifice}
\begin{itemize}
	\item Sacrifice was literal, material, \& costly.
	\Subitem Extravagance
	\Subitem Material purposes:
	\Subsubitem {{\textit{Dehí me dádāmi te} (e.g.~TS 1.8.4.1)}}
	\Subitem {\emph{Literally} carried up to the Gods}
	\Subsubitem Iṣṭā-pūrtá \& Svargá
	\Subsubitem {\textit{Svadʰā́ ca yátra tŕ̥ptiş ca / tátra mā́m amŕ̥taṁ kr̥dʰi} (R̥V 9.113.10)}
	\Subsubitem Harmful food in Nírr̥ti (AV 5.19.3, 6.63.1): blood, hair
	\Subsubitem {\textit{Nā́' sy' āmúṣmiṁ loké 'nnaṁ kṣīyate}: ``His food perishes not in yonder world.'' (TS 1.7.3.4)}
	\item Úc-cʰiṣṭa: leavings
	\Subitem Important differences from prasādá!
\end{itemize}
\end{frame}

\begin{frame} \frametitle{The food hymn: R̥V 1.187}
\begin{itemize}
	\item \textit{Pitú} (from $\surd$\textit{pi} ``swell''): nourishment (food \& drink)
	\Subitem Including sóma! cf.~R̥V 1.187.8--9
	\item \textit{Asmā́kam avitā́ bʰava}: ``Be our helper!'' (R̥V 1.187.2)
	\item \textit{Sákʰā su-şévav ádvayāḥ}: ``friend dear \& guileless''  (R̥V 1.187.3)
	\item \textit{Tué pito mahā́naam / Devā́naam máno hitám}: ``In thee, O food, is set the spirit of the great Gods.'' (R̥V 1.187.6)
	\item See also the ghee hymn, R̥V 4.58.
\end{itemize}
\end{frame}

\begin{frame} \frametitle{The Vedic Ā́rya-s' diet}
\begin{itemize}
	\item Grains: yáva (barley?), dʰānā́, sáktu, odaná
	\item Beans: mā́ṣa, mudgá
	\item Fruit (pʰála): melon (urvāruká)
	\item Aquatic plants: ávakā
	\item \& others, e.g.~sóma, alcohol (súrā, e.g.~R̥V 1.116.7)
	\item But chiefly, animal products
	\Subitem Honey (mádʰu)
	\Subitem Dairy: milk (kṣīrá, dugdʰá, páyas), curd (dadʰán), butter (sarpís), ghee (gʰr̥tá), sour cream (mástu), mixture (āmíkṣā)
	\Subitem Meat (mā́s, e.g.~R̥V 1.161.10): buffalo (mahiṣá, e.g.~R̥V 5.29.8), goat (ajá, e.g.~R̥V 1.162.4), sheep (meṣá, e.g.~R̥V 10.91.14); rarely horses (e.g.~R̥V 1.162.9) \& kine (e.g.~R̥V 10.86.14)
\end{itemize}
\end{frame}

\begin{frame} \frametitle{Dietary restrictions in the Véda-s}
\begin{itemize}
	\item Discussed at our 7$^{\text{th}}$ meeting
	\item Certainly no vegetarianism: this is Vedic literature 101
	\Subitem Interesting question: Is vegetarianism \emph{compatible} with the Vedic religion?
	\item Beef generally not permitted (\textit{ágʰnyā}), especially from cow (female)
	\Subitem But, exceptions for very special sacrifices, \& only for oxen \& barren (vaşā́) kine
	\item Horse consumption seems also to have been very rare.
	\item No reference to actual cannibalism at all despite metaphors, not even in puruṣa-medʰá
	\item Alcohol was generally allowed but seems to have been discouraged or prohibited in certain classes even in Vedic times.
	\item Other recommendations---manner/time of eating, cleanliness, \&c.?
\end{itemize}
\end{frame}

\againframe{questions}

\begin{frame} \frametitle{Some interesting papers \& articles}
\begin{itemize}
	\item Ganesh, S. ``The Hindu view on food \& drink.'' \textit{Indica Today}, \textbf{2020}.
	\item Guha, D.~S. ``Food in the Vedic tradition.'' \textit{India Int.~Cent.~Q.}, \textbf{1985}.
	\item Lopez, C. ``Food \& immortality in the Veda: a gastronomic theology?'' \textit{Electron.~J.~Vedic Studies}, \textbf{1997}.
	\item Smith, B.~K. ``Eaters, food, \& social hierarchy in ancient India.'' \textit{J.~Am.~Acad.~Relig.}, \textbf{1990}.
\end{itemize}
\end{frame}

\end{document}
