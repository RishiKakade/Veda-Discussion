\documentclass[pdf]{beamer}
\mode<presentation>{}
\usetheme{Rochester}
\usecolortheme{whale}
\beamertemplatenavigationsymbolsempty

\usepackage[ngerman]{babel}
\usepackage{fontspec}
\setmainfont[ItalicFont={brilli.ttf}, BoldFont={brillb.ttf}, Numbers=Lining]{brill.ttf}
\usefonttheme{serif}
\usepackage{microtype}
\usepackage{multicol}

\newcommand{\Subitem}[1]{{\setlength\itemindent{12pt} \item[-] #1}}
\newcommand{\Subsubitem}[1]{{\setlength\itemindent{24pt} \item[○] #1}}
\newcommand{\Subsubsubitem}[1]{{\setlength\itemindent{36pt} \item[-] #1}}
\newcommand{\Subsubsubsubitem}[1]{{\setlength\itemindent{48pt} \item[○] #1}}
\DeclareMathOperator*{\argmax}{arg\,max}

\title{Meeting 34: Şrutí, Şrāutá, Şā́kʰā}
\subtitle{Examining the Vedic corpus}
\author{Nikhil Surya Dwibhashyam}
\date{4 December 2022}

\begin{document}

\frame{\titlepage}

\begin{frame} \frametitle{Why a Véda discussion group?}
\begin{itemize}
	\item What is Vedic literature?
	\Subitem Sáṁhitā-s
	\Subitem Brā́hmaṇa-s
	\Subitem Others?
	\item Véda-s vs.~Vedā́nta, Upaniṣád-s, Purāṇá-s
	\Subitem Later monism (ádvāita) vs.~earlier dualism
	\Subitem Moral--political subtexts
	\item Modern (religious, scholarly) fixation upon later literature
\end{itemize}
\end{frame}

\begin{frame} \frametitle{What is there to discuss?}
\begin{itemize}
	\item History \& linguistics
	\item Metaphysics of religion
	\item Exegesis (e.g.~western vs.~Eastern narratives)
	\item Moral principles
	\item Nuances of Ṡrāutá ritual
	\item How we can apply Vedic principles to our lives
\end{itemize}
\end{frame}

\begin{frame} \frametitle{Our plans}
\begin{itemize}
	\item Weekly ``general audience'' meetings on Sundays at 1000 EST
	\Subitem Followed by in-depth literature meeting at 1130 EST
	\item Different topic/aspect/angle discussed every week
	\item Format:
	\Subitem Præsentation
	\Subitem Free-form discussion
	\item Eventually something more
	\Subitem Children's books
	\Subitem Poëtry collective
	\Subitem Documentation
	\Subitem Actual rituals
\end{itemize}
\end{frame}

\begin{frame} \frametitle{WhatsApp group}
\begin{center}
	\href{https://chat.whatsapp.com/IXCQEkhfrcwHI7CNY8Fgat}{https://chat.whatsapp.com/IXCQEkhfrcwHI7CNY8Fgat}
\end{center}
\end{frame}

\begin{frame} \frametitle{Let's introduce ourselves!}
\begin{itemize}
	\item Briefly!
	\item Name
	\item School (if student)
	\item How you found this group
	\item Background/interest in Hinduism \& Vedic literature
\end{itemize}
\end{frame}

\begin{frame}[label=questions] \frametitle{Some questions to consider}
\begin{itemize}
	\item What is the Vedic corpus?
	\Subitem What is Şrutí? How accurate is the label?
	\item What is the chronology of the Vedic corpus?
	\Subitem External vs.~internal chronology?
	\item What work needs to be done in collation?
	\item What new information might future discoveries reveal?
\end{itemize}
\end{frame}

\begin{frame} \frametitle{Overarching division of literature}
\begin{itemize}
	\item Şrutí vs.~Smr̥tí
	\Subitem {$\surd$şru vs.~$\surd$smr̥}
	\item Vedic vs.~non-Vedic/post-Vedic
	\item Different themes
	\item Different theologies
	\item Different ideologies?
\end{itemize}
\end{frame}

\begin{frame} \frametitle{Four layers of Şrutí}
\begin{itemize}
	\item Sáṁhitā
	\Subitem Mántra
	\item Brā́hmaṇa
	\item Not to be confused with Brāhmaṇá
	\item Āraṇyaka
	\item Upaniṣád
\end{itemize}
\end{frame}

\begin{frame} \frametitle{The Véda-s}
\begin{itemize}
	\item Ŕ̥c
	\item Yájus
	\item Sā́man
	\pause \item Átʰarvan
\end{itemize}
\end{frame}

\begin{frame} \frametitle{Şā́kʰā-s of the Ŕ̥c}
\begin{itemize}
	\item Şākala
	\Subitem Surviving S
	\Subitem B, Ā, U: Āitareya
	\item Bāṣkala
	\Subitem S, B, Ā, U: Kāuşītaka
	\item Āşvalāyana
	\Subitem S
	\Subitem B, Ā, U?
	\Subitem ŞS!
	\item Ṡāṅkʰāyana
	\Subitem No S
	\Subitem B, Ā/U
	\Subitem ŞS!
	\item Māṇḍukāyana
\end{itemize}
\end{frame}

\begin{frame} \frametitle{Şā́kʰā-s of the Yájus}
\begin{itemize}
	\item Two divisions: White and Black
	\item White:
	\Subitem Mādhyaṁdina
	\Subitem Kāṇva
	\item Black:
	\Subitem Tāittirīya
	\Subitem Māitrāyaṇi
	\Subitem Kāṭhaka
	\Subitem Kapiṣṭhala?
\end{itemize}
\end{frame}

\begin{frame} \frametitle{On apāuruṣeyatva}
\begin{itemize}
	\item Sáṁhitā
	\Subitem ``The newest hymn''
	\Subitem Poëtic innovation
	\item Brā́hmaṇa
	\Subitem Clear lāukika references
	\Subsubitem Disagreement between sages, \&c.
	\Subitem Ritual manuals rooted in world
	\item Āraṇyaka, Upaniṣád
	\Subitem Similar
\end{itemize}
\end{frame}

\againframe{questions}

\end{document}
