\documentclass[pdf]{beamer}
\mode<presentation>{}
\usetheme{Rochester}
\usecolortheme{whale}
\beamertemplatenavigationsymbolsempty

\usepackage[ngerman]{babel}
\usepackage{fontspec}
\setmainfont[ItalicFont={brilli.ttf}, BoldFont={brillb.ttf}, BoldItalicFont={brillbi.ttf}]{brill.ttf}
\usefonttheme{serif}
\usepackage{microtype}
\usepackage{multicol}

\newcommand{\Subitem}[1]{{\setlength\itemindent{12pt} \item[-] #1}}
\newcommand{\Subsubitem}[1]{{\setlength\itemindent{24pt} \item[○] #1}}
\newcommand{\Subsubsubitem}[1]{{\setlength\itemindent{36pt} \item[-] #1}}
\newcommand{\Subsubsubsubitem}[1]{{\setlength\itemindent{48pt} \item[○] #1}}

\title{Meeting 21: Die Vr̥trá-Schlacht}
\subtitle{The most famous heroïsm of Índra}
\author{Nikhil Surya Dwibhashyam}
\date{14 August 2022}

\begin{document}

\frame{\titlepage}

\begin{frame} \frametitle{Meeting agenda}
\begin{itemize}
	\item (Re)introduction to discussion group
	\item Today's topic præsentation
	\item Free discussion
\end{itemize}
\end{frame}

\begin{frame} \frametitle{Why a Véda discussion group?}
\begin{itemize}
	\item What is Vedic literature?
	\Subitem Sáṁhitā-s
	\Subitem Brā́hmaṇa-s
	\Subitem Others?
	\item Véda-s vs.~Vedā́nta \& Upaniṣád-s
	\Subitem Later monism (ádvāita) vs.~earlier dualism (dvāitá)
	\item Modern (religious, scholarly) fixation upon later literature
\end{itemize}
\end{frame}

\begin{frame} \frametitle{What is there to discuss?}
\begin{itemize}
	\item History \& linguistics
	\item Metaphysics of religion
	\item Exegesis (e.g.~countering western narratives)
	\item Moral principles
	\item Nuances of Ṡrāutá ritual
\end{itemize}
\end{frame}

\begin{frame} \frametitle{Our plans}
\begin{itemize}
	\item Weekly meetings on Sundays at 1200 EDT
	\Subitem Followed by more in-depth literature meeting at 1330 EDT
	\item Different topic/aspect/angle discussed in depth every week
	\item Format:
	\Subitem Præsentation
	\Subitem Free-form discussion
	\item Eventually something more
\end{itemize}
\end{frame}

\begin{frame} \frametitle{WhatsApp group}
\begin{center}
	\href{https://chat.whatsapp.com/IXCQEkhfrcwHI7CNY8Fgat}{https://chat.whatsapp.com/IXCQEkhfrcwHI7CNY8Fgat}
\end{center}
\end{frame}

\begin{frame} \frametitle{Let's introduce ourselves!}
\begin{itemize}
	\item Name
	\item School (if student)
	\item How you found this group
	\item Briefly: background/interest in Hinduism \& Vedic literature
\end{itemize}
\end{frame}

\begin{frame}[label=questions] \frametitle{Some questions to consider}
\begin{itemize}
	\item What are some of the ``messages'' or meanings that can be derived from the Vr̥trá myth?
	\item What parallels does the Vr̥trá myth have within the Sáṁhitā? within the Vedic religion?
	\Subitem in other IE religions? or even outside of them?
	\item What is the significance of the Vr̥trá myth to Ā́rya history and the conflict with the Dásyu-s?
	\item What traits of the Vr̥trá myth are still mysterious to us?
\end{itemize}
\end{frame}

\begin{frame} \frametitle{Who was Vr̥trá?}
\begin{itemize}
	\item Etym.: $\surd$\textit{vr̥} ``cover'' + \textit{-tra} = ``coverer''
	\item Common noun: ``enemy''
	\Subitem Sometimes translated as ``obstacle''
	\item Certainly some sort of serpent (áhi): good evidence for this within Véda
	\Subitem e.g. \textit{ābʰogá}
	\Subitem First-born among them (R̥V 1.32.4)
	\Subitem Sometimes translated by westerners as ``dragon'' in imitation of IE \& Chinese motifs
	\Subsubitem Semantics
	\item Distinct from Áhi Budʰnyà, the ``Dragon of the Deep'' (Griffith)
\end{itemize}
\end{frame}

\begin{frame} \frametitle{Who/what are the Vr̥trá-s?}
\begin{itemize}
	\item A matter of some debate
	\item Usual interpretation: common noun $\approx$ metaphorical use
	\Subitem i.e.~our foes are to us as Vr̥trá was to Índra
	\Subitem There are not multiple beings named Vr̥trá.
	\Subitem {\textit{Vr̥tráṁ Vr̥tratáram}: ``Vr̥trá worst of foes”} (R̥V 1.32.5)
	\Subitem Similar to Mitrá--mitrá
	\item Alternate interpretation: the Vr̥trá-s are a class of beings like the Rudrá-s
	\Subitem Similar to cyclical motifs in later Hinduism
\end{itemize}
\end{frame}

\begin{frame} \frametitle{What did Vr̥trá do?}
\begin{itemize}
	\item Story best narrated in præviously cited hymn, R̥V 1.32
	\item Association with mountain
	\Subitem ``Coiled around''
	\item Held back the ``waters'' or rivers
	\Subitem Sometimes interpreted as drought myth
	\Subitem ``Freeing'' of waters would be rain, in connection with storm-aspect
	\item Challenged Índra
\end{itemize}
\end{frame}

\begin{frame} \frametitle{How was Vr̥trá slain?}
\begin{itemize}
	\item The iconic deed of Índra
	\Subitem Epithet \textit{Vr̥trahán(tama)} used most commonly for Índra
	\Subsubitem But used for other Gods (Sóma, Agní, \&c.): both proper and common
	\Subitem Motif found also in Avesta: Vǝrǝθraγna
	\item Use of sóma by Índra
	\item Aided by friends/allies (e.g.~Víṣṇu)
	\item Slain by thunder-bolt (vájra) alongside forts
	\Subitem Dā́nu?
	\item Clear parallel to Vála myth (but separate)
	\item Aftermath?
\end{itemize}
\end{frame}

\againframe{questions}

\end{document}
